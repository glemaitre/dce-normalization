\section{Methods}\label{sec:methods}

\subsection{Normalization of \ac{dce}-\ac{mri} images}\label{sec:norm}

\subsection{Quantification of \ac{dce}-\ac{mri}}\label{sec:stateart}

\subsubsection{Brix and Hoffmann models}\label{sec:brixhoffmann}

In the Brix model~\citep{brix1991pharmacokinetic}, the \ac{mri} signal intensity is assumed to be proportional to the media concentration.
Therefore, the model is expressed as in Eq.\,\eqref{eq:brix}:

\begin{equation}
  s_n(t) = 1 + A \left[ \frac{\exp(k_{el} t') - 1}{k_{ep}(k_{ep} - k_{el})} \exp(- k_{el} t) - \frac{\exp(k_{ep} t') - 1}{k_{el}(k_{ep} - k_{el})} \exp(- k_{ep} t) \right],
  \label{eq:brix}
\end{equation}

\noindent with

\begin{equation}
  s_n(t) = \frac{s(t)}{S_0},
  \label{eq:enh}
\end{equation}

\noindent where $s(t)$ and $S_0$ are the \ac{mri} signal intensity at time $t$ and the average pre-contrast \ac{mri} signal intensity, respectively; $A$, $k_{el}$, and $k_{ep}$ are a constant proportional to the transfer constant, the diffusion rate constant, and the rate constant, respectively. Additionally, during the injection time $0 \leq t \leq \tau$, $t' = t$ and afterwards while $t > \tau$, $t' = \tau$.

Following this model, \citeauthor{hoffmann1995pharmacokinetic} propose the following similar model as expressed in Eq.\,\eqref{eq:hoffmann}:

\begin{equation}
  s_n(t) = 1 + \frac{A}{\tau} \left[ \frac{k_{ep} \left( \exp(k_{el} t') - 1 \right)}{k_{el}(k_{ep} - k_{el})} \exp(- k_{el} t) - \frac{\exp(k_{ep} t') - 1}{(k_{ep} - k_{el})} \exp(- k_{ep} t) \right].
  \label{eq:hoffmann}
\end{equation}

The parameters are estimated by fitting the model using non-linear least-squares optimization solved with Levenberg-Marcquardt.

\subsubsection{Tofts model}\label{sec:tofts}

The extended Tofts model is formulated as in Eq.\,\eqref{eq:exttofts}:

\begin{equation}
  C_t(t) = K_{trans} C_p(t) \Conv \exp(-k_{ep}t) + v_p C_p(t),
  \label{eq:exttofts}
\end{equation}

\noindent where $\Conv$ is the convolution operator; $C_t(t)$ and $C_p(t)$ is the concentration of contrast agent in the tissue and in the plasma, respectively; $K_{trans}$, $k_{ep}$, and $v_p$ are the volume transfer constant, the diffusion rate constant, and the plasma volume fraction, respectively.

Therefore, Tofts model requires to:
(i) detect candidate voxels from the femoral or iliac arteries and estimate a patient-based \ac{aif} signal,
(ii) convert the \ac{mri} signal intensity (i.e., \ac{aif} and dynamic signal) to a concentration, and
(iii) in the case of a population-based \ac{aif}, estimate an \ac{aif} signal.

\begin{description}
  \item[Segmentation of artery voxels and patient-based \ac{aif} estimation] The \ac{aif} signal from \ac{dce}-\ac{mri} can be manually estimated by selecting the most-enhanced voxels from the femoral or iliac arteries~\citep{meng2010comparison}.
    Few methods have been proposed to address the automated extraction of \ac{aif} signal.
    \citeauthor{chen2008automatic} filter successively the possible candidates~\citep{chen2008automatic}:
    (i) dynamic signals with small peak are rejecting by thresholding,
    (ii) voxels with a small wash-in are rejected by thresholding,
    (iii) a blob detector is used and large enough regions are kept, and
    (iv) circular and cylindricality are used to reject the last false positive.
    \citeauthor{zhu2011automated} propose an iterative method selecting voxels which best fit a gamma variate function~\citep{zhu2011automated}.
    However, it requires to compute first and second derivatives as well as maximum curvature points.
    \citeauthor{shanbhag2012generalized} propose a 4-steps algorithm~\citep{shanbhag2012generalized,fennessy2015quantitative}:
    (i) remove slices with artefacts and find the best slices based on intrinsic anatomic landmarks and enhancement characteristics,
    (ii) find the voxel candidates using the maximum enhanced voxels and a multi-label maximum entropy based thresholding algorithm,
    (iii) excluding region next to the endorectal coil, and
    (iv) selecting the best 5 candidates which meet enhancement characteristics and that are correlated.

    All the above methods are rather complex and thus we propose a method which is based on the following simple assumptions:
    (i) all possible \ac{aif} signal candidates should have a similar shape,
    (ii) an high enhancement, and
    (iii) the arteries should be almost round and within a size range.
    Therefore, each slice is clustered into regions using K-means clustering with $k=6$.
    The cluster with the highest enhancement\textemdash i.e. corresponding to the 90\textsuperscript{th} percentile of the maximum of each dynamic signal\textemdash contain the arteries and is selected.
    Finally, regions with an eccentricity smaller than $0.5$ and an area in the range of $[100, 400]$ voxels are kept.
    Additionally, to remove voxels contaminated by partial volume effect, only the $10\%$ most enhanced voxels of the possible candidates are kept as proposed by~\citep{schabel2008uncertainty} and the average signal is computed.
    A summary of the different segmentation steps is presented in Fig.\,\ref{fig:aifseg}.
    \item[Conversion of \ac{mri} signal intensity to concentration] To estimate the free parameters of the Tofts model (see Eq.\,\eqref{eq:exttofts}), the concentration $C_t(t)$ and $C_p(t)$ need to be computed from the \ac{mri} signal intensity and the \ac{aif} signal, respectively.
      This conversion is based on the equation of the FLASH sequence\textemdash see~\ref{app:signaltoconc} for details\textemdash and is formulated as in Eq.\,\eqref{eq:conv}:
      \begin{equation}
        c(t) = \frac{1}{TR \cdot r_1} \ln\left( \frac{1 - \cos \alpha \cdot S^{*}\frac{s(t)}{S_0}}{1 - S^{*}\frac{s(t)}{S_0}} \right) - \frac{R_{10}}{r_1} ,
        \label{eq:conv}
      \end{equation}
      \noindent with,
      \begin{equation}
        S^{*} = \frac{1 - \exp(- TR \cdot R_{10})}{1 - \cos \alpha \cdot \exp(- TR \cdot R_{10})} ,
        \label{eq:sstarconv}
      \end{equation}
      \noindent where $s(t)$ is the \ac{mri} signal, $S_0$ is the \ac{mri} signal prior to the injection of the contrast media, $\alpha$ is the flip angle, $TR$ is the \acf{tr}, $R_{10}$ is the pre-contrast tissue relaxation time also equal to $\frac{1}{T_{10}}$, $r_1$ is the relaxitivity coefficient of the contrast agent.

      $T_{10}$ can be estimated from the acquisition of a T$_1$ map.
      However, this modality was not part of the clinical trial in this research and the value of $T_{10}$ was fixed to \SI{1600}{\ms} for both blood and prostate as stated in the literature~\citep{fennessy2015quantitative,de2004mr,carr2011magnetic}.
      \item[Estimation of population-based \ac{aif}] While estimating the pharmacokinetic parameters from Tofts model, the \ac{aif} concentration $C_p(t)$ can be computed either from the patient or a population.
        We presented in the two previous sections the algorithms which allows to estimate the patient-based \ac{aif} concentration.
        To compare with the previous approach, we also computed a population-based \ac{aif} which will be also later used to compare the performance of both approaches.
        In that regard, the population-based \ac{aif} was estimated as in~\citep{meng2010comparison} by fitting the average patient-based \ac{aif}s to the model of~\cite{parker2006experimentally} which is formulated as in Eq.\,\eqref{eq:parker}:
        \begin{equation}
          C_p(t) = \sum_{n=1}^{2} \frac{A_n}{\sigma_n \sqrt{2 \pi}} \exp\left(\frac{- (t- T_n)^2}{2\sigma_{n}^{2}}\right) + \frac{\alpha \exp(-\beta t)}{1 + \exp{-s (t - \tau)}} ,
          \label{eq:parker}
        \end{equation}
        \noindent where $A_n$, $T_n$, and $\sigma_n$ are the scaling constants, centers, and widths of the n\textsuperscript{th} Gaussian, $\alpha$ and $\beta$ are the amplitude and decay constant of the exponential; and $s$ and $\tau$ are the width and center of the sigmoid function, respectively.
\end{description}

The parameters are estimated by fitting the model using non-linear least-squares optimization solved with Levenberg-Marcquardt.

\subsubsection{\acs*{pun} model}\label{sec:pun}

\citeauthor{gliozzi2011phenomenological} show that \ac{pun} approach can be used for \ac{dce}-\ac{mri} analysis~\citep{gliozzi2011phenomenological}.
The model has been successfully used in a \ac{cad} system proposed by~\cite{giannini2015fully}.
This model can be expressed as in Eq.\,\eqref{eq:pun}:

\begin{equation}
  s_n(t) = \exp\left[rt + \frac{1}{\beta} \left( a_0 - r \right) \left( \exp(\beta t) - 1 \right) \right],
  \label{eq:pun}
\end{equation}

\noindent with

\begin{equation}
  s_n(t) = \frac{s(t) - S_0}{S_0},
  \label{eq:enh}
\end{equation}

\noindent where $s(t)$ and $S_0$ are the \ac{mri} signal intensity at time $t$ and the average pre-contrast \ac{mri} signal intensity, respectively; $r$, $a_0$, and $\beta$ are the free parameters of the model.

The parameters are estimated by fitting the model using non-linear least-squares optimization solved with Levenberg-Marcquardt.

\subsubsection{Semi-quantitative analysis}\label{sec:semi}

The semi-quantitative analysis of the \ac{dce}-\ac{mri} is equivalent to extract curve characteristics directly from the signal without a strict theoretical pharmacokinetic meaning.
In this work, we use the model presented by~\cite{huisman2001accurate} which formulate the \ac{mri} signal as in Eq.\,\eqref{eq:huisman}:

\begin{equation}
  s(t) = \begin{cases}
    S_0 & 0 \leq t \leq t_0 \\
    S_M - (S_M - S_0) \exp\left( \frac{-(t - t_0)}{\tau} \right) & t_0 < t \leq t_0 + 2 \tau \\
    S_M - (S_M - S_0) \exp\left( \frac{-(t - t_0)}{\tau} \right) + w(t - t_0 + 2 \tau) & t > t_0 + 2 \tau
  \end{cases}
  \label{eq:huisman}
\end{equation}

\noindent where $s(t)$ is the \ac{mri} signal intensity, $S_0$ is the pre-contrast signal intensity, $t_0$ is the time corresponding to the start of enhancement, $S_M$ and $\tau$ is the maximum of the signal and the exponential time constant, and $w$ is the slope of the linear part.

\citeauthor{huisman2001accurate} argue that curve fitting via least-squares minimization using Nelder-Mead algorithm leads to inaccurate estimation of the free parameters: mainly the issue come from an incorrect estimation of the start of enhancement $t_0$ leading to incorrect estimation of the other parameters.
Therefore, they propose to:
(i) estimate robustly $t_0$,
(ii) estimate $S_0$ by averaging the samples between $0$ and $t_0$
(ii) estimate $w$ depending if the slope is significant or not,
(iii) estimate $S_M$ which should be the point at the intersection of the most probable slope line and the plateau.

Instead of these successive estimations, we propose a unified optimization in which $t_0$ is fixed since that this is a key parameter.
Therefore, $t_0$ is robustly estimated from the \ac{aif} signal since that this is the most enhanced signal in which the start of enhancement is easily identifiable.
The \ac{aif} signal is computed as in Section~\ref{sec:tofts}.
$t_0$ is estimated by finding the maximum in the beginning of the first derivative of the \ac{mri} signal.
Then, the function in Eq.\,\eqref{eq:huisman} is fitted using non-linear least squares with Trust Region Reflective algorithm.
Furthermore, the parameters $\tau$ and $S_M$ are bounded during the optimization to ensure robust estimations.

From Eq.\,\eqref{eq:huisman}, the following features are extracted:
(i) the wash-in corresponding to the slope between $t_0$ and $t_0 + 2 \tau$,
(ii) the wash-out corresponding to the parameter $w$,
(iii) the area under the curve between $t_0$ and the end of the signal,
(iv) the exponential time constant $\tau$, and
(v) the relative enhancement $S_M - S_0$.

%%% Local Variables: 
%%% mode: latex
%%% TeX-master: "../main"
%%% End: 
