\section{Material and Methods}\label{sec:method}

\subsection{\ac{mri} data}\label{sec:data}

The multi-parametric \ac{mri} data are acquired from a cohort of patients with higher-than-normal level of \ac{psa}.
The acquisition is performed using a 3T whole body \ac{mri} scanner (Siemens Magnetom Trio TIM, Erlangen, Germany) using sequences to obtain \ac{t2w}-\ac{mri}, \ac{dce}-\ac{mri} and \ac{dw}-\ac{mri}.
Aside of the \ac{mri} examination, these patients also have undergone a guided-biopsy.
The dataset is composed of a total of 20 patients of which 18 patients have biopsy proven \ac{cap} and 2 patients are ``healthy'' with negative biopsies.
Therefore, 13 patients have a \ac{cap} in the \ac{pz}, 3 patients have \ac{cap} in the \ac{cg}, 2 patients have invasive \ac{cap} in both \ac{pz} and \ac{cg} and finally 2 patients are considered as ``healthy''.
An experienced radiologist has segmented the prostate organ --- on \ac{t2w}-\ac{mri} and \ac{dce}-\ac{mri} --- as well as the prostate zones (i.e., \ac{pz} and \ac{cg}) and \ac{cap} on the \ac{t2w}-\ac{mri}.

A \SI{3}{\mm} slice fat-suppressed \ac{t2w} fast spin-echo sequence (\ac{tr}/\ac{te}/\ac{etl}: \SI{3400}{\ms}/\SI{85}{\ms}/13) is used to acquire images in sagittal and oblique coronal planes, the latter planes being orientated perpendicular or parallel to the prostate \ac{pz} – rectal wall axis.
Three-dimensional \ac{t2w} fast spin-echo (\ac{tr}/\ac{te}/\ac{etl}: \SI{3600}{\ms}/\SI{143}{\ms}/109, slice thickness: \SI{1.25}{\mm}) images are then acquired in an oblique axial plane.
The nominal matrix and \ac{fov} of the 3D \ac{t2w} fast spin-echo images are $320 \times 256$ and $280 \times 240$ mm\textsuperscript{2}, respectively, thereby affording sub-millimetric pixel resolution within the imaging plane.

\ac{dce}-\ac{mri} is performed using a fat suppressed 3D T$_1$ VIBE sequence (\ac{tr}/\ac{te}/Flip angle: \SI{3.25}{\ms}/\SI{1.12}{\ms}/\SI{10}{\degree}; Matrix: $256 \times 192$; \ac{fov}: $280 \times 210$ (with 75\% rectangular \ac{fov}); slab of 16 partitions of \SI{3.5}{\mm} thickness; temporal resolution: \SI{6}{\s}/slab over approximately \SI{5}{\minute}).
A power injector (Medrad, Indianola, USA) is used to provide a bolus injection of Gd-DTPA (Dotarem, Guerbet, Roissy, France) at a dose of \SI{0.2}{\ml} Gd-DTPA/kg of body weight.

These \ac{dce}-\ac{mri} sequences are resampled using the spatial information of the \ac{t2w}-\ac{mri} and missing data are interpolated using a linear interpolation.
The volumes of the \ac{dce}-\ac{mri} dynamic are rigidly registered, to remove any patient motion during the acquisition.
Furthermore, a non-rigid registration is performed between the \ac{t2w}-\ac{mri} and \ac{dce}-\ac{mri} in order to propagate the prostate zones and \ac{cap} ground-truths.

The resampling is implemented in C++ using the Insight Segmentation and Registration Toolkit~\citep{ibanez2005itk}.
The implementation of the registration (C++), normalization (Python), and classification pipeline (Python) are publicly available on GitHub\footnote{\url{https://github.com/I2Cvb/lemaitre-2016-nov/tree/master}}~\citep{lemaitre2016github}.
The data used for this work are also publicly available\footnote{\url{https://zenodo.org/record/61163}}~\citep{lemaitre2016dce}.


\subsection{Normalization of \ac{dce}-\ac{mri} images}\label{sec:norm}

In this work, we propose a method to normalize \ac{dce}-\ac{mri} prostate data to reduce inter-patient variations, although it can be applied to any \ac{dce}-\ac{mri} sequences.
In \ac{t2w}-\ac{mri}, these variations are characterized by a shift and a scaling of the intensities as illustrated by the intensity \ac{pdf} in Fig.\,\ref{fig:t2w}.
Therefore, these variations can be corrected using a $z$-score approach--- i.e., normalizing the data by subtracting the mean and dividing by the standard deviation ---assuming that the data follow a specific distribution~\citep{lemaitre2016normalization}.

In \ac{dce}-\ac{mri}, the intensity \ac{pdf} of prostate gland does not follow a unique type of distribution such as Rician or Gaussian distribution, as shown in Fig.\,\ref{subfig:pathhist}.
Indeed, the inter-patient variations are more complex due to the temporal acquisition.
A better representation to observe these variations is to represent the intensity \ac{pdf} of the prostate gland over time--- requiring to segment the prostate ---using a heatmap representation as shown in Fig.\,\ref{subfig:pathhist}.
Analyzing this heatmap representation across patients (see Fig.\,\ref{subfig:pat2}), the following variations are highlighted:
(i) intensity offsets $\Delta_i$ of the \ac{pdf} peak,
(ii) a time offset $\Delta_t$ depending of the contrast agent arrival, and
(iii) a change of scale $\alpha_i$ related to the signal enhancement.
Therefore, our normalization method should attenuate all these variations and be performed globally across the different time sequences rather than for each independent sequence.

\subsubsection{Graph-based intensity offsets correction}\label{sec:intoffsets}

Before to standardize each sequence, the first step of the normalization is to cancel the intensity specific at each patient, occurring due to the media injection.
As previously mentioned, the intensity \ac{pdf} does not always follow either a Rician or a Gaussian distribution over time, in \ac{dce}-\ac{mri}.
Therefore, the mean of these distributions cannot be used as a potential estimate for these offsets.
Additionally, these offsets should be characterized by a smooth transition between series over time.
Thus, this problem is solved using the graph-theory: considering the intensity \ac{pdf} over time as shown in Fig.\,\ref{subfig:pathhist}, the offsets correspond to the boundary splitting the heatmap in two partitions such that they are as close as possible to the peak of the intensity \ac{pdf} (see Fig.\,\ref{fig:estimator} for an illustration).
Given the heatmap, a directed weighted graph $\mathcal{G}=(\mathcal{V}, \mathcal{E})$ is built by taking each bar--- i.e., the probability for a given time and pixel intensity---of the heatmap as a node and connecting each pair of bars by an edge.
The edge weight $w_{ij}$ between two nodes $i$ and $j$ corresponding to two pixels at position $(x_i, y_i)$ and $(x_j, y_j)$, respectively, is defined as in Eq.\,\eqref{eq:weight}:

\begin{equation}
  w_{ij} = \begin{cases}
    \alpha \exp(1 - \frac{H(i)}{\max(H)})       & \text{if } x_j = x_i + 1 \text{ and } y_j = y_i, \\
    (1 - \alpha) \exp(1 - \frac{H(i)}{\max(H)}) & \text{if } x_j = x_i \text{ and } y_j = y_i + 1, \\
    0                                           & \text{otherwise},
  \end{cases}
  \label{eq:weight}
\end{equation}

\noindent where $H$ is the heatmap, $\alpha$ is a smoothing parameter controlling the partitioning.

Therefore, these offsets related to $\Delta_i$ are estimated by finding the shortest-path to cross the graph using Dijkstra's algorithm.
The entry and exiting nodes are set to be the bin with the maximum probability for the first \ac{dce}-\ac{mri} serie and the bin corresponding to the median value for the last \ac{dce}-\ac{mri} serie, respectively.
To ensure a robust estimation of these offsets, the process of finding the shortest-path is iteratively repeated by shifting the data and updating the heatmap as well as the graph $\mathcal{G}$.
The procedure is stopped once the offset found does not change.
In general, this process is not repeated more than 3 iterations.
The parameter $\alpha$ is set to $0.9$, empirically.
Figure~\ref{fig:estimator} illustrates the final estimation of the offsets $\Delta_i$ (i.e., red landmark) found for each \ac{dce}-\ac{mri} serie.
Therefore, each intensity offset is subtracted for each \ac{dce}-\ac{mri}.

\subsubsection{Time offset and data dispersion correction}

The next variations to correct are the time offset $\Delta_t$ and the data dispersion $\sigma_i$.
By computing the \ac{rmse} of the intensities for each \ac{dce}-\ac{mri} serie, one can observe these two variations as shown in Fig.\,\ref{fig:rmse}.
Therefore, to correct these variations, we propose to register each patient \ac{rmse} to a mean model which corresponds to the mean of all patients \ac{rmse}.
The parametric model to perform the registration is formulated as in Eq.\,\eqref{eq:model}:

\begin{equation}
  T(\alpha, \tau, f(t)) = \alpha f(t - \tau) ,
  \label{eq:model}
\end{equation}

\noindent where $\alpha$ and $\tau$ are the two parameters handling the time offset $\Delta_i$ and global scale $\sigma_i$, respectively, $f(\cdot)$ is the \ac{rmse} function define as:

\begin{equation}
  f(t) = \sqrt{ \left( \frac{\sum_{n=1}^{N} x(t)_{n}^2}{N}  \right) },
  \label{eq:rmsd}
\end{equation}

\noindent where $x(t)_n$ is the shifted intensity of a sample from a specific \ac{dce}-\ac{mri} serie at time $t$ from a total number of $N$ samples.

Therefore the registration problem is equivalent to:

\begin{equation}
  \argmin_{\alpha, \tau} = \sum_{t=1}^{N} \left[ T\left(\alpha, \tau, f(t)\right) - \mu(t) \right]^{2} ,
  \label{eq:cost}
\end{equation}

\noindent where $\mu(\cdot)$ is the mean model, $N$ is the number of \ac{dce}-\ac{mri} serie.

Illustration of the correction applied to each \ac{rmse} patient is shown in Fig.\,\ref{fig:rmseal}.
Once all these parameters have been inferred, the data are shifted as well as scaled.

The resulting normalized data can be used into two fashions: (i) each normalized signal can be used as a whole to determine whether the corresponding voxel is healthy or cancerous or (ii) the normalized data can be fitted using a quantitative method, as presented in the next section.
%However, for the second strategy, this is necessary to apply common intensity offsets such that the data follow a shape as expected by the different quantitative models.
%The set of offsets applied is in fact corresponding to the maximum offsets found in Sect.\,\ref{sec:intoffsets}.

\subsection{Quantification of \ac{dce}-\ac{mri}}\label{sec:stateart}

In this section, we summarize the different methods which have been used for the quantification of \ac{dce}-\ac{mri} for \ac{cap} detection~\citep{lemaitre2015computer} and which will be used for comparison in this work.
Furthermore, we would like to emphasize the following additional contributions for this section: (i) a novel automatic \ac{aif} estimation algorithm based on clustering and (ii) a simplified semi-quantitative method using constrained optimization.

\subsubsection{Brix and Hoffmann models}\label{sec:brixhoffmann}

In the Brix model~\citep{brix1991pharmacokinetic}, the \ac{mri} signal intensity is assumed to be proportional to the media concentration.
Therefore, the model is expressed as in Eq.\,\eqref{eq:brix}:

\begin{equation}
  s_n(t) = 1 + A \left[ \frac{\exp(k_{el} t') - 1}{k_{ep}(k_{ep} - k_{el})} \exp(- k_{el} t) - \frac{\exp(k_{ep} t') - 1}{k_{el}(k_{ep} - k_{el})} \exp(- k_{ep} t) \right],
  \label{eq:brix}
\end{equation}

\noindent with

\begin{equation}
  s_n(t) = \frac{s(t)}{S_0},
  \label{eq:enh}
\end{equation}

\noindent where $s(t)$ and $S_0$ are the \ac{mri} signal intensity at time $t$ and the average pre-contrast \ac{mri} signal intensity, respectively; $A$, $k_{el}$, and $k_{ep}$ are the constant proportional to the transfer constant, the diffusion rate constant, and the rate constant, respectively.
Additionally, $t'$ is set such that $0 \leq t \leq \tau$, $t' = t$ and afterwards while $t > \tau$, $t' = \tau$.

\citeauthor{hoffmann1995pharmacokinetic} proposed a similar model as expressed in Eq.\,\eqref{eq:hoffmann}, which derive from the Brix model:

\begin{equation}
  \small
  s_n(t) = 1 + \frac{A}{\tau} \left[ \frac{k_{ep} \left( \exp(k_{el} t') - 1 \right)}{k_{el}(k_{ep} - k_{el})} \exp(- k_{el} t) - \frac{\exp(k_{ep} t') - 1}{(k_{ep} - k_{el})} \exp(- k_{ep} t) \right] ,
  \label{eq:hoffmann}
\end{equation}

\noindent in which the constant $A$ is redefined by isolating the parameter $\tau$.

The parameters $A$, $k_{el}$, and $k_{ep}$ are estimated by fitting the model using non-linear least-squares optimization solved with Levenberg-Marcquardt.

\subsubsection{Tofts model}\label{sec:tofts}

The extended Tofts model is formulated as in Eq.\,\eqref{eq:exttofts}:

\begin{equation}
  C_t(t) = K_{trans} C_p(t) \Conv \exp(-k_{ep}t) + v_p C_p(t),
  \label{eq:exttofts}
\end{equation}

\noindent where $\Conv$ is the convolution operator; $C_t(t)$ and $C_p(t)$ are the concentrations of contrast agent in the tissue and in the plasma, respectively; $K_{trans}$, $k_{ep}$, and $v_p$ are the volume transfer constant, the diffusion rate constant, and the plasma volume fraction, respectively.

Therefore, Tofts model requires to:
(i) detect candidate voxels from the femoral or iliac arteries and estimate a patient-based \ac{aif} signal,
(ii) convert the \ac{mri} signal intensity (i.e., \ac{aif} and dynamic signal) to a concentration, and
(iii) in the case of a population-based \ac{aif}, estimate an \ac{aif} signal.

\begin{description}
  \item[Segmentation of artery voxels and patient-based \ac{aif} estimation] The \ac{aif} signal from \ac{dce}-\ac{mri} can be manually estimated by selecting the most-enhanced voxels from the femoral or iliac arteries~\citep{meng2010comparison}.
    Few methods have been proposed to address the automated extraction of \ac{aif} signal.
    \citeauthor{chen2008automatic} filtered successively the possible candidates to be considered as \ac{aif} such that~\citep{chen2008automatic}:
    (i) dynamic signals with small peak and voxels with a small wash-in are rejected by thresholding,
    (ii) a blob detector is used and large enough regions are kept, and
    (iii) circular and cylindricality criteria are used to reject the false positives.
    \citeauthor{zhu2011automated} proposed an iterative method selecting voxels which best fit a gamma variate function~\citep{zhu2011automated}.
    However, it requires to compute first and second derivatives as well as maximum curvature points.
    \citeauthor{shanbhag2012generalized} proposed a 4-steps algorithm~\citep{shanbhag2012generalized,fennessy2015quantitative}:
    (i) remove slices with artefacts and find the best slices based on intrinsic anatomic landmarks and enhancement characteristics,
    (ii) find the voxel candidates using the maximum enhanced voxels and a multi-label maximum entropy based thresholding algorithm,
    (iii) exclude region next to the endorectal coil, and
    (iv) select the best 5 candidates which meet enhancement characteristics and that are correlated.

    All the above methods are rather complex and thus we propose a simpler method which is based on the following reasonable assumptions:
    (i) all possible \ac{aif} signal candidates should have a similar shape,
    (ii) a high enhancement, and
    (iii) the arteries should be almost round and within a size range.
    Therefore, each slice is clustered into regions using K-means clustering with $k=6$.
    The cluster made of the most enhanced signals is selected since it contains the artery signals.
    In this regards, the selection criteria corresponds to the 90\textsuperscript{th} percentile of the maximum \ac{dce}-\ac{mri} signal.
    Finally, regions with an eccentricity smaller than $0.5$ and an area in the range of $[100, 400]$ voxels are kept.
    Additionally, to remove voxels contaminated by partial volume effect, only the $10\%$ most enhanced voxels of the possible candidates are kept as proposed by~\citep{schabel2008uncertainty} and the average signal is computed.
    A summary of the different segmentation steps is presented in Fig.\,\ref{fig:aif}.
    \item[Conversion of \ac{mri} signal intensity to concentration] To estimate the free parameters of the Tofts model (see Eq.\,\eqref{eq:exttofts}), the concentration $C_t(t)$ and $C_p(t)$ need to be computed from the \ac{mri} signal intensity and the \ac{aif} signal, respectively.
      This conversion is based on the equation of the FLASH sequence\textemdash see Appendix~\ref{app:signaltoconc} for details\textemdash and is formulated as in Eq.\,\eqref{eq:conv}:
      \begin{equation}
        c(t) = \frac{1}{TR \cdot r_1} \ln\left( \frac{1 - \cos \alpha \cdot S^{*}\frac{s(t)}{S_0}}{1 - S^{*}\frac{s(t)}{S_0}} \right) - \frac{R_{10}}{r_1} ,
        \label{eq:conv}
      \end{equation}
      \noindent with,
      \begin{equation}
        S^{*} = \frac{1 - \exp(- TR \cdot R_{10})}{1 - \cos \alpha \cdot \exp(- TR \cdot R_{10})} ,
        \label{eq:sstarconv}
      \end{equation}
      \noindent where $s(t)$ is the \ac{mri} signal, $S_0$ is the \ac{mri} signal prior to the injection of the contrast media, $\alpha$ is the flip angle, $TR$ is the \acf{tr}, $R_{10}$ is the pre-contrast tissue relaxation time also equal to $\frac{1}{T_{10}}$, and $r_1$ is the relaxitivity coefficient of the contrast agent.

      $T_{10}$ can be estimated from the acquisition of a T$_1$ map.
      However, this modality is not part of the clinical trial in this research and the value of $T_{10}$ is fixed to \SI{1600}{\ms} for both blood and prostate, in accordance with the values found in the literature~\citep{fennessy2015quantitative,de2004mr,carr2011magnetic}.
      \item[Estimation of population-based \ac{aif}] While estimating the pharmacokinetic parameters from Tofts model, the \ac{aif} concentration $C_p(t)$ can be computed either from the patient or a population.
        We presented in the two previous sections the algorithms which allows to estimate the patient-based \ac{aif} concentration.
        To compare with the previous approach, we also computed a population-based \ac{aif} which will be also used later to compare the performance of both approaches.
        In that regard, the population-based \ac{aif} was estimated as in~\citep{meng2010comparison} by fitting the average patient-based \ac{aif}s to the model of~\cite{parker2006experimentally} which is formulated as in Eq.\,\eqref{eq:parker}:
        \begin{equation}
          C_p(t) = \sum_{n=1}^{2} \frac{A_n}{\sigma_n \sqrt{2 \pi}} \exp\left(\frac{- (t- T_n)^2}{2\sigma_{n}^{2}}\right) + \frac{\alpha \exp(-\beta t)}{1 + \exp{-s (t - \tau)}} ,
          \label{eq:parker}
        \end{equation}
        \noindent where $A_n$, $T_n$, and $\sigma_n$ are the scaling constants, centers, and widths of the n\textsuperscript{th} Gaussian, $\alpha$ and $\beta$ are the amplitude and decay constant of the exponential; and $s$ and $\tau$ are the width and center of the sigmoid function, respectively.
\end{description}

The parameters are estimated by fitting the model using a constrained non-linear least-squares optimization, solved with the Trust Region Reflective algorithm~\citep{sorensen1982newton} and bounding the parameters to be positive.

\subsubsection{\acs*{pun} model}\label{sec:pun}

\citeauthor{gliozzi2011phenomenological} showed that \ac{pun} approach can be used for \ac{dce}-\ac{mri} analysis~\citep{gliozzi2011phenomenological}.
The model has been successfully used in a \ac{cad} system proposed by~\cite{giannini2015fully}.
This model can be expressed as in Eq.\,\eqref{eq:pun}:

\begin{equation}
  s_n(t) = \exp\left[rt + \frac{1}{\beta} \left( a_0 - r \right) \left( \exp(\beta t) - 1 \right) \right],
  \label{eq:pun}
\end{equation}

\noindent with

\begin{equation}
  s_n(t) = \frac{s(t) - S_0}{S_0},
  \label{eq:enh}
\end{equation}

\noindent where $s(t)$ and $S_0$ are the \ac{mri} signal intensity at time $t$ and the average pre-contrast \ac{mri} signal intensity, respectively; $r$, $a_0$, and $\beta$ are the free parameters of the model.

The parameters are estimated by fitting the model using non-linear least-squares optimization solved with Levenberg-Marcquardt.

\subsubsection{Semi-quantitative analysis}\label{sec:semi}

The semi-quantitative analysis of the \ac{dce}-\ac{mri} is equivalent to extracting curve characteristics directly from the signal without a strict theoretical pharmacokinetic meaning.
In this work, we use the model presented by~\cite{huisman2001accurate} which formulated the \ac{mri} signal as in Eq.\,\eqref{eq:huisman}:

\begin{equation}
  s(t) = \begin{cases}
    S_0 & 0 \leq t \leq t_0 \\
    S_M - (S_M - S_0) \exp\left( \frac{-(t - t_0)}{\tau} \right) & t_0 < t \leq t_0 + 2 \tau \\
    S_M - (S_M - S_0) \exp\left( \frac{-(t - t_0)}{\tau} \right) + w(t - t_0 + 2 \tau) & t > t_0 + 2 \tau
  \end{cases}
  \label{eq:huisman}
\end{equation}

\noindent where $s(t)$ is the \ac{mri} signal intensity, $S_0$ is the pre-contrast signal intensity, $t_0$ is the time corresponding to the start of enhancement, $S_M$ and $\tau$ is the maximum of the signal and the exponential time constant, and $w$ is the slope of the linear part.

\citeauthor{huisman2001accurate} argue that curve fitting via least-squares minimization using Nelder-Mead algorithm leads to inaccurate estimation of the free parameters: mainly the issue come from an incorrect estimation of the start of enhancement $t_0$ leading to incorrect estimation of the other parameters.
Therefore, they propose to:
(i) estimate robustly $t_0$,
(ii) estimate $S_0$ by averaging the samples between $0$ and $t_0$
(ii) estimate $w$ depending if the slope is significant or not,
(iii) estimate $S_M$ which should be the point at the intersection of the most probable slope line and the plateau.

Instead of these successive estimations, we propose a unified optimization in which $t_0$ is fixed since that this is a key parameter.
Therefore, $t_0$ is robustly estimated from the \ac{aif} signal since that this is the most enhanced signal in which the start of enhancement is easily identifiable.
The \ac{aif} signal is computed as in Section~\ref{sec:tofts}.
$t_0$ is estimated by finding the maximum of the first derivative of the \ac{aif} signal, always occurring at the beginning of the signal.
Then, the function in Eq.\,\eqref{eq:huisman} is fitted using non-linear least squares with the Trust Region Reflective algorithm~\citep{sorensen1982newton}.
Furthermore, the parameters $\tau$ and $S_M$ are bounded during the optimization to ensure robust estimations.
$\tau$ is bounded between $t_0$ and $t_f$ which is the time of the last sample while $S_M$ is bounded between $S_0$ and $\max(s(t))$.


From Eq.\,\eqref{eq:huisman}, the following features are extracted:
(i) the wash-in corresponding to the slope between $t_0$ and $t_0 + 2 \tau$,
(ii) the wash-out corresponding to the parameter $w$,
(iii) the area under the curve between $t_0$ and the end of the signal,
(iv) the exponential time constant $\tau$, and
(v) the relative enhancement $S_M - S_0$.

%%% Local Variables: 
%%% mode: latex
%%% TeX-master: "../main"
%%% End: 
