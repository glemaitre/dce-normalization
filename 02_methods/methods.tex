\section{Methods}\label{sec:methods}

\subsection{Normalization of \ac{dce}-\ac{mri} images}\label{sec:norm}

\subsection{Quantification of \ac{dce}-\ac{mri}}\label{sec:stateart}

\subsubsection{Brix and Hoffmann models}

In the Brix model~\citep{brix1991pharmacokinetic}, the \ac{mri} signal intensity is assumed to be proportional to the media concentration.
Therefore, the model is expressed as in Eq.\,\eqref{eq:brix}:

\begin{equation}
  s_n(t) = 1 + A \left[ \frac{\exp(k_{el} t') - 1}{k_{ep}(k_{ep} - k_{el})} \exp(- k_{el} t) - \frac{\exp(k_{ep} t') - 1}{k_{el}(k_{ep} - k_{el})} \exp(- k_{ep} t) \right],
  \label{eq:brix}
\end{equation}

\noindent with

\begin{equation}
  s_n(t) = \frac{s(t)}{S_0},
  \label{eq:enh}
\end{equation}

\noindent where $s(t)$ and $S_0$ are the \ac{mri} signal intensity at time $t$ and the average pre-contrast \ac{mri} signal intensity, respectively; $A$, $k_{el}$, and $k_{ep}$ are a constant proportional to the transfer constant, the diffusion rate constant, and the rate constant, respectively. Additionally, during the injection time $0 \leq t \leq \tau$, $t' = t$ and afterwards while $t > \tau$, $t' = \tau$.

Following this model, \citeauthor{hoffmann1995pharmacokinetic} propose the following similar model as expressed in Eq.\,\eqref{eq:hoffmann}:

\begin{equation}
  s_n(t) = 1 + \frac{A}{\tau} \left[ \frac{k_{ep} \left( \exp(k_{el} t') - 1 \right)}{k_{el}(k_{ep} - k_{el})} \exp(- k_{el} t) - \frac{\exp(k_{ep} t') - 1}{(k_{ep} - k_{el})} \exp(- k_{ep} t) \right].
  \label{eq:hoffmann}
\end{equation}

\subsubsection{Tofts model}

The extended Tofts model is formulated as in Eq.\,\eqref{eq:exttofts}:

\begin{equation}
  C_t(t) = K_{trans} C_p(t) \Conv \exp(-k_{ep}t) + v_p C_p(t),
  \label{eq:exttofts}
\end{equation}

\noindent where $\Conv$ is the convolution operator; $C_t(t)$ and $C_p(t)$ is the concentration of contrast agent in the tissue and in the plasma, respectively; $K_{trans}$, $k_{ep}$, and $v_p$ are the volume transfer constant, the diffusion rate constant, and the plasma volume fraction, respectively.

Therefore, Tofts model requires to:
(i) detect candidate voxels from the femoral or iliac arteries and estimate a patient-based \ac{aif} signal,
(ii) convert the \ac{mri} signal intensity (i.e., \ac{aif} and dynamic signal) to a concentration, and
(iii) in the case of a population-based \ac{aif}, estimate an \ac{aif} signal.

\begin{description}
  \item[Segmentation of artery voxels and patient-based \ac{aif} estimation] The \ac{aif} signal from \ac{dce}-\ac{mri} can be manually estimated by selecting the most-enhanced voxels from the femoral or iliac arteries~\citep{meng2010comparison}.
    Few methods have been proposed to address the automated extraction of \ac{aif} signal.
    \citeauthor{chen2008automatic} filter successively the possible candidates~\citep{chen2008automatic}:
    (i) dynamic signals with small peak are rejecting by thresholding,
    (ii) voxels with a small wash-in are rejected by thresholding,
    (iii) a blob detector is used and large enough regions are kept, and
    (iv) circular and cylindricality are used to reject the last false positive.
    \citeauthor{zhu2011automated} propose an iterative method selecting voxels which best fit a gamma variate function~\citep{zhu2011automated}.
    However, it requires to compute first and second derivatives as well as maximum curvature points.
    \citeauthor{shanbhag2012generalized} propose a 4-steps algorithm~\citep{shanbhag2012generalized,fennessy2015quantitative}:
    (i) remove slices with artefacts and find the best slices based on intrinsic anatomic landmarks and enhancement characteristics,
    (ii) find the voxel candidates using the maximum enhanced voxels and a multi-label maximum entropy based thresholding algorithm,
    (iii) excluding region next to the endorectal coil, and
    (iv) selecting the best 5 candidates which meet enhancement characteristics and that are correlated.

    All the above methods are rather complex and thus we propose a method which is based on the following simple assumptions:
    (i) all possible \ac{aif} signal candidates should have a similar shape,
    (ii) an high enhancement, and
    (iii) the arteries should be almost round and within a size range.
    Therefore, each slice is clustered into regions using K-means clustering with $k=6$.
    The cluster with the highest enhancement\textemdash i.e. corresponding to the 90\textsuperscript{th} percentile of the maximum of each dynamic signal\textemdash contain the arteries and is selected.
    Finally, regions with an eccentricity smaller than $0.5$ and an area in the range of $[100, 400]$ voxels are kept.
    Additionally, to remove voxels contaminated by partial volume effect, only the $10\%$ most enhanced voxels of the possible candidates are kept as proposed by~\citep{schabel2008uncertainty}.
    An example of artery segmentation is shown in Fig.\,\ref{fig:aifseg}.
    \item[Conversion of \ac{mri} signal intensity to concentration]    
\end{description}

\subsubsection{\acs*{pun} model}

\citeauthor{gliozzi2011phenomenological} show that \ac{pun} approach can be used for \ac{dce}-\ac{mri} analysis~\citep{gliozzi2011phenomenological}.
The model has been successfully used in a \ac{cad} system proposed by~\cite{giannini2015fully}.
This model can be expressed as in Eq.\,\eqref{eq:pun}:

\begin{equation}
  s_n(t) = \exp\left[rt + \frac{1}{\beta} \left( a_0 - r \right) \left( \exp(\beta t) - 1 \right) \right],
  \label{eq:pun}
\end{equation}

\noindent with

\begin{equation}
  s_n(t) = \frac{s(t) - S_0}{S_0},
  \label{eq:enh}
\end{equation}

\noindent where $s(t)$ and $S_0$ are the \ac{mri} signal intensity at time $t$ and the average pre-contrast \ac{mri} signal intensity, respectively; $r$, $a_0$, and $\beta$ are the free parameters of the model.

\subsubsection{}

%%% Local Variables: 
%%% mode: latex
%%% TeX-master: "../main"
%%% End: 
