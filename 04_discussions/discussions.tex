\section{Discussions}\label{sec:discussions}

The experiments conducted in the previous section can incite several discussions.
In Tofts quantification, two different approaches have been used to infer the pharmacokinetic parameters: using a population-based or a patient-based \ac{aif}.
The patient-based \ac{aif} approach leads to better classification performance.
However, there are two shortcomings to take into account when considering this fact:
(i) the T$_{10}$ parameter has been fixed and is not computed from a T$_1$ map and
(ii) the population-based \ac{aif} has been estimated from a cohort of only 17 patients.
These two limitations have to be considered when asserting that
population-based \ac{aif} modeling outperforms patient-based \ac{aif} modeling.

The best classification performance is achieved by normalizing the
\ac{dce}-\ac{mri} data and using the entire enhanced signal as a
feature, emphasizing the fact that a \emph{loss of information} may occur while extracting quantitative parameters.
Furthermore, normalization is a less complex process than all
quantification methods and significantly improves the classification
performance of the semi-quantitative approach ($p=0.001$) proposed by
\citeauthor{huisman2001accurate}.
Additionally, our normalization improves the \ac{auc} score related to
the detection of \ac{cap} in \ac{cg}, also known to be the most
challenging cases in the diagnosis of \ac{cap}.

However, using the entire enhanced signal in conjunction with the
normalization is limited by one drawback: the training time of the
\ac{rf} classifier increases; instead of using $3$ to $5$ features, so the
feature space becomes a $40$ dimensional space. The extraction of the
semi-quantitative parameters leads to a comparable classification performance ---
$0.655 \pm 0.108$ vs. $0.666 \pm 0.154$ --- and should be chosen as the alternative method to
consider if the number of feature in the classification is
critical. Despite this fact, the benefit of our proposed normalization method
has been shown for both methods.

Nevertheless, this study is performed on a small cohort of patients using a single \ac{mri} machine.
Generalizing the results of this study onto a larger dataset acquired
from different commercial systems needs to be considered to study the robustness of the proposed approach.
Additionally, our method being non-parametric should provide an
adequate framework to process \ac{dce} data from different
institutions, scanners with different settings.

%%% Local Variables:
%%% mode: latex
%%% TeX-master: "../main"
%%% End:
