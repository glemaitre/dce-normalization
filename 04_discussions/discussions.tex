\section{Discussions}\label{sec:discussions}

The experiments conducted in the previous section can give rise to several discussions.
In Tofts quantification, two different approaches have been used to infer the pharmacokinetic parameters: using a population-based or a patient-based \ac{aif}.
The population-based \ac{aif} approach leads to better classification performance.
However, there is two shortcomings to take into account while advancing this fact:
(i) T$_{10}$ parameter has been fixed and not computed from a T$_1$ map and
(ii) the population-based \ac{aif} has been estimated from a cohort of only 17 patients.
These two limitations have to be considered while advancing that population-based \ac{aif} modelling is outperforming patient-based \ac{aif} modelling.

The best classification performance is reached by normalizing the \ac{dce}-\ac{mri} data and use the whole enhanced signal as feature, emphasizing the fact that a loss of information while extracting quantitative parameters.
However, this strategy suffers from one drawback: the training time of the \ac{rf} classifier increases since that from 3 to 5 features, the feature space become a 40 dimensions space.

%%% Local Variables: 
%%% mode: latex
%%% TeX-master: "../main"
%%% End: 
