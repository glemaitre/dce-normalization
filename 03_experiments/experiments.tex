\section{Materials}\label{sec:materials}

\subsection{Data}\label{sec:data}

The multi-parametric \ac{mri} data are acquired from a cohort of patients with higher-than-normal level of \ac{psa}.
The acquisition is performed using a 3T whole body \ac{mri} scanner (Siemens Magnetom Trio TIM, Erlangen, Germany) using sequences to obtain \ac{t2w}-\ac{mri}, \ac{dce}-\ac{mri} and \ac{dw}-\ac{mri}.
Aside of the \ac{mri} examination, these patients also have undergone a guided-biopsy.
The dataset is composed of a total of 20 patients of which 18 patients have biopsy proven \ac{cap} and 2 patients are ``healthy'' with negative biopsies.
Therefore, 13 patients have a \ac{cap} in the \ac{pz}, 3 patients have \ac{cap} in the \ac{cg}, 2 patients have invasive \ac{cap} in both \ac{pz} and \ac{cg} and finally 2 patients are considered as ``healthy''.
An experienced radiologist has segmented the prostate organ --- on \ac{t2w}-\ac{mri} and \ac{dce}-\ac{mri} --- as well as the prostate zones (i.e., \ac{pz} and \ac{cg}) and \ac{cap} on the \ac{t2w}-\ac{mri}.

A \SI{3}{\mm} slice fat-suppressed \ac{t2w} fast spin-echo sequence (\ac{tr}/\ac{te}/\ac{etl}: \SI{3400}{\ms}/\SI{85}{\ms}/13) is used to acquire images in sagittal and oblique coronal planes, the latter planes being orientated perpendicular or parallel to the prostate \ac{pz} – rectal wall axis.
Three-dimensional \ac{t2w} fast spin-echo (\ac{tr}/\ac{te}/\ac{etl}: \SI{3600}{\ms}/\SI{143}{\ms}/109, slice thickness: \SI{1.25}{\mm}) images are then acquired in an oblique axial plane.
The nominal matrix and \ac{fov} of the 3D \ac{t2w} fast spin-echo images are $320 \times 256$ and $280 \times 240$ mm\textsuperscript{2}, respectively, thereby affording sub-millimetric pixel resolution within the imaging plane.

\ac{dce}-\ac{mri} is performed using a fat suppressed 3D T$_1$ VIBE sequence (\ac{tr}/\ac{te}/Flip angle: \SI{3.25}{\ms}/\SI{1.12}{\ms}/\SI{10}{\degree}; Matrix: $256 \times 192$; \ac{fov}: $280 \times 210$ (with 75\% rectangular \ac{fov}); slab of 16 partitions of \SI{3.5}{\mm} thickness; temporal resolution: \SI{6}{\s}/slab over approximately \SI{5}{\minute}).
A power injector (Medrad, Indianola, USA) is used to provide a bolus injection of Gd-DTPA (Dotarem, Guerbet, Roissy, France) at a dose of \SI{0.2}{\ml} Gd-DTPA/kg of body weight.

These \ac{dce}-\ac{mri} sequences are resampled using the spatial information of the \ac{t2w}-\ac{mri} and missing data are interpolated using a linear interpolation.
The volumes of the \ac{dce}-\ac{mri} dynamic are rigidly registered, to remove any patient motion during the acquisition.
Furthermore, a non-rigid registration is performed between the \ac{t2w}-\ac{mri} and \ac{dce}-\ac{mri} in order to propagate the prostate zones and \ac{cap} ground-truths.
The resampling is implemented in C++ using the Insight Segmentation and Registration Toolkit~\citep{ibanez2005itk}.

\subsection{Implementation}

The implementation of the registration (C++), normalization (Python), and classification pipeline (Python) are publicly available on GitHub\footnote{\url{https://github.com/I2Cvb/lemaitre-2016-nov/tree/master}}~\citep{lemaitre2016github}.
The data used for this work are also publicly available\footnote{\url{https://zenodo.org/record/61163}}~\citep{lemaitre2016dce}.

\section{Experiments and results}\label{sec:experiments}

\subsection{Goodness of model fitting}\label{sec:fit}

%{\color{red} In case that we have issue with $R^2$, we need to provide the AIC since that the model are usually non-linear.}

\begin{table*}
  \caption{Coefficient of determination $R^{2}$ (i.e., $\mu \ (\pm \sigma)$), while fitting data with the different quantification models.}
  \centering
  \resizebox{\columnwidth}{!}{
  \begin{tabular}{lc c c c c c}
    \toprule
    Data type & Brix & Hoffmann & Tofts population \ac{aif} & Tofts patient \ac{aif} & \ac{pun} & Semi-quantitative \\
    \midrule
    Un-normalized & $0.85 \ (\pm 0.11)$ & $0.81 \ (\pm 0.17)$ & $0.84 \ (\pm 0.14)$ & $0.88 \ (\pm 0.12)$ & $0.27 \ (\pm 0.18)$ & $0.64 \ (\pm 0.24)$  \\
    Normalized    & $0.92 \ (\pm 0.05)$ & $0.72 \ (\pm 0.32)$ & $0.92 \ (\pm 0.06)$ & $0.90 \ (\pm 0.10)$ & $0.28 \ (\pm 0.20)$ & $0.75 \ (\pm 0.20)$  \\
    \bottomrule
  \end{tabular}
  }
  \label{tab:r2}
\end{table*}

Parameter estimation of the quantification methods are related to fit a specific model to the \ac{dce}-\ac{mri} data.
Therefore, this section reports the goodness of fitting by computing the coefficient of determination $R^2$ such as in Eq.\,\eqref{eq:r2}

\begin{equation}
  R^2 = 1 - \frac{\sum_{t = 1}^{T} (s_t - \hat{s}_t)^2}{\sum_{t = 1}^{T} (s_t - \bar{s})^2} ,
  \label{eq:r2}
\end{equation}

\noindent where $s_t$ and $\hat{s}_t$ are the signal to be fitted and the estimated signal at time $t$, respectively; $\bar{s}$ is the average signal to be fitted.

Mean and standard-deviation of the coefficient of determination $R^{2}$ is reported in Table~\ref{tab:r2} for each quantification model.
Brix, Hoffmann, and Tofts models are fitted with a coefficient $R^{2}$ superior to 0.80.
Additionally, the proposed \ac{pun} model does not to fit well the data.
After introspection of the fitted curves, the original model --- as
formulated in Eq.\,\eqref{eq:pun} --- does not provide enough degrees
of freedom to fit well the data. Additional parameters should be
integrated in this model to control the translation and amplitude of
the model to be fitted.
Data normalization improves the coefficient $R^2$ for all the methods apart of the Hoffmann model.
The large standard deviation for this model might imply that there are some cases where the fitting fails.

% \subsection{Detection of \acs*{cap} using pharmacokinetic parameters}

\subsection{\acs*{cap} detection using pharcokinetic,
  semi-quantitative, and entire enhanced signal}

\begin{table*}
  \caption{\acs*{auc} (i.e., $\mu \ (\pm \sigma)$) for each individual pharmacokinetic parameter using a \acs*{rf} classifier.}
  \centering
  \resizebox{\columnwidth}{!}{
  \begin{tabular}{l c c}
    \toprule
    \textbf{Features} & Un-normalized data & Normalized data \\
    \midrule
    \textbf{Brix model} & & \\
    \quad $A$         & $0.540\ (\pm 0.069)$ & $0.555\ (\pm 0.080)$ \\
    \quad $k_{el}$    & $0.549\ (\pm 0.062)$ & $0.577\ (\pm 0.093)$ \\
    \quad $k_{ep}$    & $0.506\ (\pm 0.032)$ & $0.497\ (\pm 0.019)$ \\
    \textbf{Hoffmann model} & & \\
    \quad $A$         & $0.516\ (\pm 0.020)$ & $0.508\ (\pm 0.031)$ \\
    \quad $k_{el}$    & $0.545\ (\pm 0.066)$ & $0.529\ (\pm 0.065)$ \\
    \quad $k_{ep}$    & $0.550\ (\pm 0.063)$ & $0.545\ (\pm 0.060)$ \\
    \textbf{Tofts model with population \ac{aif}} & & \\
    \quad $K_{trans}$ & $0.556\ (\pm 0.086)$ & $0.565\ (\pm 0.097)$ \\
    \quad $k_{ep}$    & $0.506\ (\pm 0.026)$ & $0.528\ (\pm 0.038)$ \\
    \quad $v_{p}$     & $0.533\ (\pm 0.064)$ & $0.548\ (\pm 0.082)$ \\
    \textbf{Tofts model with patient \ac{aif}} & & \\
    \quad $K_{trans}$ & $0.563\ (\pm 0.077)$ & $0.548\ (\pm 0.060)$ \\
    \quad $k_{ep}$    & $0.492\ (\pm 0.025)$ & $0.491\ (\pm 0.020)$ \\
    \quad $v_{p}$     & $0.530\ (\pm 0.069)$ & $0.495\ (\pm 0.033)$ \\
    \textbf{\ac{pun} model} & & \\
    \quad $a_0$       & $0.521\ (\pm 0.040)$ & $0.530\ (\pm 0.045)$ \\
    \quad $r$         & $0.550\ (\pm 0.085)$ & $0.573\ (\pm 0.097)$ \\
    \quad $\beta$     & $0.531\ (\pm 0.051)$ & $0.549\ (\pm 0.068)$ \\
    \textbf{Semi-quantitative analysis} & & \\
    \quad wash-in     & $0.587\ (\pm 0.107)$ & $0.533\ (\pm 0.032)$ \\
    \quad wash-out    & $0.516\ (\pm 0.037)$ & $0.486\ (\pm 0.035)$ \\
    \quad IAUC        & $0.506\ (\pm 0.048)$ & $0.513\ (\pm 0.032)$ \\
    \quad $\tau$      & $0.565\ (\pm 0.104)$ & $0.537\ (\pm 0.089)$ \\
    \quad $S_M - S_0$ & $0.560\ (\pm 0.083)$ & $0.532\ (\pm 0.029)$ \\
    \bottomrule
  \end{tabular}
  }
  \label{tab:resfeats}
\end{table*}

\begin{figure*}
  \centering
  \hspace*{\fill}
  \subfigure[Quantiative model without normalization.]{\label{fig:rfpharmaunorm}\includegraphics[width=.49\textwidth]{03_experiments/figures/unormalized/unormalized_methods_0.pdf}} \hfill
  \subfigure[Quantitative model With normalization.]{\label{fig:rfpharmanorm}\includegraphics[width=.49\textwidth]{03_experiments/figures/normalized/normalized_methods_0.pdf}}
  \hspace*{\fill}
  \hspace*{\fill}
  \subfigure[Entire enhanced \acs*{dce} signal classification.]{\label{fig:rfnormdcesignal}\includegraphics[width=.49\textwidth]{03_experiments/figures/full_signal_0.pdf}}
  \hspace*{\fill}
  \caption{\acs*{roc} analysis using a \acs*{rf} classifier with and
    without normalization using different approaches: \protect\subref{fig:rfpharmaunorm} -
    \protect\subref{fig:rfpharmanorm} pharmacokinetic and
    semi-quantitative models without and with normalization,
    respectively; \protect\subref{fig:rfnormdcesignal} entire enhanced
  \acs*{dce}-\acs*{mri} signal.}
  \label{fig:normpharmarf}
\end{figure*}

\begin{figure}
  \centering
  \includegraphics[width=1.\linewidth]{03_experiments/figures/wilcoxon.pdf}
  \caption{Wilcoxon signed-ranked test to compare each pair of
    classifiers. The annotation in the matrix corresponds to the
    p-values ($p$). $p < 0.05$ indicates that a pair of classifiers
    are significantly different, consequently the classifier with the
    highest \acs*{auc} significantly outperforms the other
    one. In the co-occurrence matrix, {\color{blue}blue} cells
    correspond to lower $p$ while {\color{red} red} cells are
    representing greater $p$.}
  \label{fig:wilcoxon}
\end{figure}

To study the potential benefit of our normalization, \ac{cap} are detected at a voxel level using pharmacokinetic parameters estimated from un-normalized and normalized \ac{dce}-\ac{mri} data.
Each individual pharmacokinetic parameter is classified to evaluate their individual discriminative power to detect \ac{cap}.
Therefore, a \ac{rf} classifier is used in conjunction with a \ac{lopo}.
The use of \ac{rf} is motivated since that it leads to the best performance in the state-of-the-art methods~\citep{litjens2014computer,lemaitre2015computer}.
Results are summarized in Table~\ref{tab:resfeats} in terms of \ac{auc}.
Normalization can improve the detection of \ac{cap}; however, the benefit of normalization is more obvious by combining together the pharmacokinetic features of a given model (e.g., $A$, $k_ep$, and $k_el$ for Brix model), as previously done in traditional \ac{cad} system~\citep{lemaitre2015computer}.
For the latter configuration, results are summarized by performing a \ac{roc} analysis and computing the \ac{auc}, as reported in Fig.\,\ref{fig:normpharmarf}.
Quantification using normalized data outperforms quantification using un-normalized data in terms of classification performance apart of Hoffmann and Tofts population-based \ac{aif} models.
The reasons behind the decrease of the \ac{auc} might be related to: (i) a poor fitting as discussed in Sect.\,\ref{sec:fit} (cf., Hoffmann model) and (ii) a small number of patients while estimating some parameters (cf., Tofts model).
The best classification performance are obtained using the semi-quantitative approach with an \ac{auc} of $0.655$.

%\subsection{Classification of the entire enhanced \acs*{dce}-\acs*{mri} signal}

\begin{figure*}
  \centering
  \includegraphics[width=1.\linewidth]{03_experiments/figures/plot_auc_by_zones.pdf}
  \caption{Comparison of the distribution of the \acs*{auc} score for
    each prostate zone, with and without normalization.}
  \label{fig:boxplot}
\end{figure*}

As stated in the introduction, the quantification methods are extracting a set of parameters characterizing the enhancement \ac{dce}-\ac{mri} signal.
However, this extraction might lead to a loss of information.
This experiment is performed to assess if making use of the whole \ac{dce}-\ac{mri} signal instead of the just the pharmacokinetic parameters can improve the classification performance.
Therefore, each enhanced \ac{dce}-\ac{mri} signal, normalized and un-normalized, is classified using a \ac{rf} classifier in a \ac{lopo} fashion.
The \ac{roc} analysis and \ac{auc} are reported in Fig.\,\ref{fig:rfnormdcesignal}.
Classification without normalization lead to the worst performance, with an \ac{auc} of $0.568$.
However, data normalization in conjunction with the use of the entire \ac{dce}-\ac{mri} signal is the strategy which outperforms others, with an \ac{auc} of $0.666$.
Comparable outcomes are visible while analyzing the \ac{auc}
distributions scores for each prostate zone. The \ac{auc} score
related to \ac{cap} in \ac{cg} improves using normalized data, for
most of the models.

%%% Local Variables:
%%% mode: latex
%%% TeX-master: "../main"
%%% End:
