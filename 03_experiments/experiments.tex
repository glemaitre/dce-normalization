\section{Materials}\label{sec:materials}

\subsection{Data}\label{sec:data}

The multi-parametric \ac{mri} data are acquired from a cohort of patients with higher-than-normal level of \ac{psa}.
The acquisition is performed using a 3T whole body \ac{mri} scanner (Siemens Magnetom Trio TIM, Erlangen, Germany) using sequences to obtain \ac{t2w}-\ac{mri}, \ac{dce}-\ac{mri} and \ac{dw}-\ac{mri}.
Aside of the \ac{mri} examination, these patients also have underwent a guided-biopsy.
The dataset is composed of a total of 20 patients of which 18 patients have biopsy proven \ac{cap} and 2 patients are ``healthy'' with negative biopsies.
Therefore, 13 patients have a \ac{cap} in the \ac{pz}, 3 patients have \ac{cap} in the \ac{cg}, 2 patients have invasive \ac{cap} in both \ac{pz} and \ac{cg} and finally 2 patients are considered as ``healthy''.
An experienced radiologist has segmented the prostate organ --- on \ac{t2w}-\ac{mri} and \ac{dce}-\ac{mri} --- as well as the prostate zones (i.e., \ac{pz} and \ac{cg}) and \ac{cap} on the \ac{t2w}-\ac{mri}.

A \SI{3}{\mm} slice fat-suppressed \ac{t2w} fast spin-echo sequence (\ac{tr}/\ac{te}/\ac{etl}: \SI{3400}{\ms}/\SI{85}{\ms}/13) is used to acquire images in sagittal and oblique coronal planes, the latter planes being orientated perpendicular or parallel to the prostate \ac{pz} – rectal wall axis.
Three-dimensional \ac{t2w} fast spin-echo (\ac{tr}/\ac{te}/\ac{etl}: \SI{3600}{\ms}/\SI{143}{\ms}/109, slice thickness: \SI{1.25}{\mm}) images are then acquired in an oblique axial plane.
The nominal matrix and \ac{fov} of the 3D \ac{t2w} fast spin-echo images are $320 \times 256$ and $280 \times 240$ mm\textsuperscript{2}, respectively, thereby affording sub-millimetric pixel resolution within the imaging plane.

\ac{dce}-\ac{mri} is performed using a fat suppressed 3D T$_1$ VIBE sequence (\ac{tr}/\ac{te}/Flip angle: \SI{3.25}{\ms}/\SI{1.12}{\ms}/\SI{10}{\degree}; Matrix: $256 \times 192$; \ac{fov}: $280 \times 210$ (with 75\% rectangular \ac{fov}); slab of 16 partitions of \SI{3.5}{\mm} thickness; temporal resolution: \SI{6}{\s}/slab over approximately \SI{5}{\minute}).
A power injector (Medrad, Indianola, USA) is used to provide a bolus injection of Gd-DTPA (Dotarem, Guerbet, Roissy, France) at a dose of \SI{0.2}{\ml} Gd-DTPA/kg of body weight.

These \ac{dce}-\ac{mri} sequences are resampled using the spatial information of the \ac{t2w}-\ac{mri} and missing data are interpolated using a linear interpolation.
The volumes of the \ac{dce}-\ac{mri} dynamic are rigidly registered, to remove any patient motion during the acquisition.
Furthermore, a non-rigid registration is performed between the \ac{t2w}-\ac{mri} and \ac{dce}-\ac{mri} in order to propagate the prostate zones and \ac{cap} ground-truths.
The resampling is implemented in C++ using the Insight Segmentation and Registration Toolkit~\citep{ibanez2005itk}.

\subsection{Implementation}

The implementation of the registration (C++), normalization (Python), and classification pipeline (Python) are publicly available on GitHub\footnote{\url{https://github.com/I2Cvb/lemaitre-2016-nov/tree/master}}.
The data used for this work are also publicly available\footnote{\url{http://kaggle.com}}.

\section{Experiments and results}\label{sec:experiments}

\subsection{Classification of individual parameter for each model}

% \begin{table}
%   \caption{\acs*{auc} of the individual features for each method.}
%   \centering
%   \begin{tabular}{l c c c c}
%     \toprule
%     \textbf{Features} & \multicolumn{2}{c}{Un-normalized data} & \multicolumn{2}{c}{Normalized data} \\
%     & \acs*{rf} & \acs*{nb} & \acs*{rf} & \acs*{nb} \\
%     \midrule
%     \textbf{Brix model} & & & & \\
%     \quad $A$         & 0.54 & 0.62 & 0.58 & 0.67 \\
%     \quad $k_{el}$    & 0.55 & 0.52 & 0.54 & 0.61 \\
%     \quad $k_{ep}$    & 0.51 & 0.52 & 0.51 & 0.58 \\
%     \textbf{Hoffmann model} & & & & \\
%     \quad $A$         & 0.52 & 0.50 & 0.51 & 0.56 \\
%     \quad $k_{el}$    & 0.55 & 0.53 & 0.54 & 0.64 \\
%     \quad $k_{ep}$    & 0.55 & 0.50 & 0.53 & 0.66 \\
%     \textbf{Tofts model with population \ac{aif}} & & & & \\
%     \quad $K_{trans}$ & 0.56 & 0.62 & 0.56 & 0.65 \\
%     \quad $v_{e}$     & 0.51 & 0.50 & 0.50 & 0.52 \\
%     \quad $v_{p}$     & 0.53 & 0.63 & 0.55 & 0.53 \\
%     \textbf{Tofts model with patient \ac{aif}} & & & & \\
%     \quad $K_{trans}$ & 0.57 & 0.66 & 0.56 & 0.65 \\
%     \quad $v_{e}$     & 0.49 & 0.50 & 0.51 & 0.52 \\
%     \quad $v_{p}$     & 0.53 & 0.37 & 0.57 & 0.65 \\
%     \textbf{\ac{pun} model} & & & & \\
%     \quad $a_0$       & 0.52 & 0.53 & 0.53 & 0.51  \\
%     \quad $r$         & 0.53 & 0.59 & 0.55 & 0.55 \\
%     \quad $\beta$     & 0.55 & 0.56 & 0.53 & 0.44 \\
%     \textbf{Semi-quantitative analysis} & & & & \\
%     \quad wash-in     & 0.59 & 0.64 & 0.55 & 0.51 \\
%     \quad wash-out    & 0.52 & 0.50 & 0.56 & 0.66 \\
%     \quad IAUC        & 0.51 & 0.61 & 0.52 & 0.64 \\
%     \quad $\tau$      & 0.57 & 0.57 & 0.56 & 0.61 \\
%     \quad $S_M - S_0$ & 0.56 & 0.63 & 0.53 & 0.64 \\
%     \bottomrule
%   \end{tabular}
%   \label{tab:resfeats}
% \end{table}

\begin{table*}
  \caption{\acs*{auc} for each individual pharmacokinetic parameter using a \acs*{nb} classifier.}
  \centering
  \begin{tabular}{l c c c}
    \toprule
    \textbf{Features} & Un-normalized data & Normalized data & Gain/Loss \\
    \midrule
    \textbf{Brix model} & & & \\
    \quad $A$         & 0.62 & 0.67 & + 0.05 \\
    \quad $k_{el}$    & 0.52 & 0.61 & + 0.09 \\
    \quad $k_{ep}$    & 0.52 & 0.58 & + 0.06 \\
    \textbf{Hoffmann model} & & & \\
    \quad $A$         & 0.50 & 0.56 & + 0.06 \\
    \quad $k_{el}$    & 0.53 & 0.64 & + 0.11 \\
    \quad $k_{ep}$    & 0.50 & 0.66 & + 0.16 \\
    \textbf{Tofts model with population \ac{aif}} & & & \\
    \quad $K_{trans}$ & 0.62 & 0.65 & + 0.03 \\
    \quad $v_{e}$     & 0.50 & 0.52 & + 0.02 \\
    \quad $v_{p}$     & 0.63 & 0.53 & - 0.10 \\
    \textbf{Tofts model with patient \ac{aif}} & & & \\
    \quad $K_{trans}$ & 0.66 & 0.65 & - 0.01 \\
    \quad $v_{e}$     & 0.50 & 0.52 & + 0.02 \\
    \quad $v_{p}$     & 0.37 & 0.65 & + 0.18 \\
    \textbf{\ac{pun} model} & & & \\
    \quad $a_0$       & 0.53 & 0.51 & - 0.02 \\
    \quad $r$         & 0.59 & 0.55 & - 0.04 \\
    \quad $\beta$     & 0.56 & 0.44 & - 0.12 \\
    \textbf{Semi-quantitative analysis} & & & \\
    \quad wash-in     & 0.64 & 0.51 & - 0.15 \\
    \quad wash-out    & 0.50 & 0.66 & + 0.16 \\
    \quad IAUC        & 0.61 & 0.64 & + 0.03 \\
    \quad $\tau$      & 0.57 & 0.61 & + 0.04 \\
    \quad $S_M - S_0$ & 0.63 & 0.64 & + 0.01 \\
    \bottomrule
  \end{tabular}
  \label{tab:resfeats}
\end{table*}

This experiment consists in assessing the benefit of normalizing the \ac{dce}-\ac{mri}, prior to classify each individual parameter for each model.
Therefore, a Gaussian \ac{nb} classifier is used in conjunction with a \ac{lopo}.
The results are summarized in Table~\ref{tab:resfeats}.
In general, data normalization leads to better classification performance (i.e., + 4\%) while extracting pharmacokinetic related parameters.
Only the \ac{pun} model does not follow this tendency.
{\color{red} Find out why!!! I am computing the R2 coefficient to check the fit of data.}

\subsection{Classification by combining the parameters for each model}

\begin{figure*}
  \centering
  \hspace*{\fill}
  \subfigure[Without normalization.]{\label{fig:rfpharmaunorm}\includegraphics[width=.49\textwidth]{03_experiments/figures/unormalized/rf.pdf}} \hfill
  \subfigure[With normalization.]{\label{fig:rfpharmanorm}\includegraphics[width=.49\textwidth]{03_experiments/figures/normalized/rf.pdf}}
  \hspace*{\fill}
  \caption{\acs*{roc} analysis using a \acs*{rf} classifier with and without normalization \ac{dce}-\ac{mri} data for different pharmacokinetic models.}
  \label{fig:normpharmarf}
\end{figure*}

In previous proposed \ac{cad} systems, the pharmacokinetic of a single model are combined.
Therefore, for each model, all parameters are combined and classified using a \ac{rf} classifier in a \ac{lopo} fashion.
The use of \ac{rf} is motivated since that it leads to the best performance in the state-of-the-art methods~\citep{litjens2014computer}.
Results are summarized by performing a \ac{roc} analysis and computing the \ac{auc}, as reported in Fig.\,\ref{fig:normpharmarf}.
Independently to the quantification model, the data normalization improves the\ac{auc} and classification performance.
An increase of the \ac{auc} ranging from 1\% to 12\% is noticeable.

% \begin{figure}
%   \centering
%   \hspace*{\fill}
%   \subfigure[Without normalization.]{\label{fig:nbpharmaunorm}\includegraphics[width=.49\textwidth]{03_experiments/figures/unormalized/nb.pdf}} \hfill
%   \subfigure[With normalization.]{\label{fig:nbpharmanorm}\includegraphics[width=.49\textwidth]{03_experiments/figures/normalized/nb.pdf}}
%   \hspace*{\fill}
%   \caption{\acs*{roc} analysis using a \acs*{nb} classifier with and without normalization \ac{dce}-\ac{mri} data for different pharmacokinetic models.}
%   \label{fig:normpharmanb}
% \end{figure}

% \begin{figure}
%   \centering
%   \hspace*{\fill}
%   \subfigure[\acs*{rf} classifier.]{\label{fig:rfunorm}\includegraphics[width=.49\textwidth]{03_experiments/figures/rf.pdf}} \hfill
%   \subfigure[\acs*{nb} classifier.]{\label{fig:nbunorm}\includegraphics[width=.49\textwidth]{03_experiments/figures/nb.pdf}}
%   \hspace*{\fill}
%   \caption{\acs*{roc} analysis using the entire \ac{dce}-\ac{mri} signal with and without normalization.}
%   \label{fig:unormdcesignal}
% \end{figure}

\subsection{Classification of the entire enhanced \acs*{dce}-\acs*{mri} signal}

\begin{figure}
  \centering
  \includegraphics[width=0.7\linewidth]{03_experiments/figures/rf.pdf}
  \caption{\acs*{roc} analysis using the entire \ac{dce}-\ac{mri} signal with and without normalization in conjunction with a \acs*{rf} classifier.}
  \label{fig:rfnormdcesignal}
\end{figure}

As stated in the introduction, the quantification methods are extracting a set of parameters characterizing the enhancement \ac{dce}-\ac{mri} signal.
However, this extraction could lead to a loss of information.
This experiment is performed to assess if making use of the whole \ac{dce}-\ac{mri} signal instead of the just the pharmacokinetic parameters can improve the classification performance.
Therefore, each enhanced \ac{dce}-\ac{mri} signal, normalized and un-normalized, is classified using a \ac{rf} classifier in a \ac{lopo} fashion.
The \ac{roc} analysis and \ac{auc} are reported in Fig.\,\ref{fig:rfnormdcesignal}.
Classification without normalization lead to the worst performance, with an \ac{auc} of 0.57.
However, data normalization in conjunction with the use of the whole \ac{dce}-\ac{mri} signal is the strategy which outperforms all others, with an \ac{auc} of 0.67.

%%% Local Variables: 
%%% mode: latex
%%% TeX-master: "../main"
%%% End: 
