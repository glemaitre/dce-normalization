%\section{Materials}\label{sec:materials}

\section{Results}\label{sec:experiments}

\subsection{Goodness of model fitting}\label{sec:fit}

%{\color{red} In case that we have issue with $R^2$, we need to provide the AIC since that the model are usually non-linear.}

Parameter estimation of the quantification methods are related to fit a specific model to the \ac{dce}-\ac{mri} data.
Therefore, this section report the goodness of fitting by computing the coefficient of determination $R^2$ such as in Eq.\,\eqref{eq:r2}

\begin{equation}
  R^2 = 1 - \frac{\sum_{t = 1}^{T} (s_t - \hat{s}_t)^2}{\sum_{t = 1}^{T} (s_t - \bar{s})^2} ,
  \label{eq:r2}
\end{equation}

\noindent where $s_t$ and $\hat{s}_t$ are the signal to be fitted and the estimated signal at time $t$, respectively; $\bar{s}$ is the average signal to be fitted.

Mean and standard-deviation of the coefficient of determination $R^{2}$ is reported in Table~\ref{tab:r2} for each quantification model.
Brix, Hoffmann, and Tofts models are fitted with a coefficient $R^{2}$ superior to 0.80.
Additionally, the proposed \ac{pun} model does not seem to fit well the data.
Data normalization improves the coefficient $R^2$ for all the methods apart of the Hoffmann model.
The large standard deviation for this model might imply that there are some cases where the fitting fails.

\subsection{Detection of \acs*{cap} using pharmacokinetic parameters}

To study the potential benefit of our normalization, \ac{cap} are detected at a voxel level using pharmacokinetic parameters estimated from un-normalized and normalized \ac{dce}-\ac{mri} data.
Each individual pharmacokinetic parameter is classified to evaluate their individual discriminative power to detect \ac{cap}.
Therefore, a \ac{rf} classifier is used in conjunction with a \ac{lopo}.
The use of \ac{rf} is motivated since that it leads to the best performance in the state-of-the-art methods~\citep{litjens2014computer,lemaitre2015computer}.
Results are summarized in Table~\ref{tab:resfeats} in terms of \ac{auc}.
Normalization can improve the detection of \ac{cap}; however, the benefit of normalization is more obvious by combining together the pharmacokinetic features of a given model (e.g., $A$, $k_ep$, and $k_el$ for Brix model), as previously done in traditional \ac{cad} system~\citep{lemaitre2015computer}.
For the latter configuration, results are summarized by performing a \ac{roc} analysis and computing the \ac{auc}, as reported in Fig.\,\ref{fig:normpharmarf}.
Quantification using normalized data outperforms quantification using un-normalized data in terms of classification performance apart of Hoffmann and Tofts population-based \ac{aif} models.
The reasons behind the decrease of the \ac{auc} might be related to: (i) a poor fitting as discussed in Sect.\,\ref{sec:fit} (cf., Hoffmann model) and (ii) a small number of patients while estimating some parameters (cf., Tofts model).
The best classification performance are obtained using the semi-quantitative approach with an \ac{auc} of 0.655.

\subsection{Classification of the entire enhanced \acs*{dce}-\acs*{mri} signal}

As stated in the introduction, the quantification methods are extracting a set of parameters characterizing the enhancement \ac{dce}-\ac{mri} signal.
However, this extraction might lead to a loss of information.
This experiment is performed to assess if making use of the whole \ac{dce}-\ac{mri} signal instead of the just the pharmacokinetic parameters can improve the classification performance.
Therefore, each enhanced \ac{dce}-\ac{mri} signal, normalized and un-normalized, is classified using a \ac{rf} classifier in a \ac{lopo} fashion.
The \ac{roc} analysis and \ac{auc} are reported in Fig.\,\ref{fig:rfnormdcesignal}.
Classification without normalization lead to the worst performance, with an \ac{auc} of 0.568.
However, data normalization in conjunction with the use of the whole \ac{dce}-\ac{mri} signal is the strategy which outperforms all others, with an \ac{auc} of 0.666.

%%% Local Variables: 
%%% mode: latex
%%% TeX-master: "../main"
%%% End: 
