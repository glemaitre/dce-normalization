\section{Materials}\label{sec:materials}

\subsection{Data}\label{sec:data}

The multi-parametric \ac{mri} data are acquired from a cohort of patients with higher-than-normal level of \ac{psa}.
The acquisition is performed using a 3T whole body \ac{mri} scanner (Siemens Magnetom Trio TIM, Erlangen, Germany) using sequences to obtain \ac{t2w}-\ac{mri}, \ac{dce}-\ac{mri} and \ac{dw}-\ac{mri}.
Aside of the \ac{mri} examination, these patients also have undergone a guided-biopsy.
The dataset is composed of a total of 20 patients of which 18 patients have biopsy proven \ac{cap} and 2 patients are ``healthy'' with negative biopsies.
Therefore, 13 patients have a \ac{cap} in the \ac{pz}, 3 patients have \ac{cap} in the \ac{cg}, 2 patients have invasive \ac{cap} in both \ac{pz} and \ac{cg} and finally 2 patients are considered as ``healthy''.
An experienced radiologist has segmented the prostate organ --- on \ac{t2w}-\ac{mri} and \ac{dce}-\ac{mri} --- as well as the prostate zones (i.e., \ac{pz} and \ac{cg}) and \ac{cap} on the \ac{t2w}-\ac{mri}.

A \SI{3}{\mm} slice fat-suppressed \ac{t2w} fast spin-echo sequence (\ac{tr}/\ac{te}/\ac{etl}: \SI{3400}{\ms}/\SI{85}{\ms}/13) is used to acquire images in sagittal and oblique coronal planes, the latter planes being orientated perpendicular or parallel to the prostate \ac{pz} – rectal wall axis.
Three-dimensional \ac{t2w} fast spin-echo (\ac{tr}/\ac{te}/\ac{etl}: \SI{3600}{\ms}/\SI{143}{\ms}/109, slice thickness: \SI{1.25}{\mm}) images are then acquired in an oblique axial plane.
The nominal matrix and \ac{fov} of the 3D \ac{t2w} fast spin-echo images are $320 \times 256$ and $280 \times 240$ mm\textsuperscript{2}, respectively, thereby affording sub-millimetric pixel resolution within the imaging plane.

\ac{dce}-\ac{mri} is performed using a fat suppressed 3D T$_1$ VIBE sequence (\ac{tr}/\ac{te}/Flip angle: \SI{3.25}{\ms}/\SI{1.12}{\ms}/\SI{10}{\degree}; Matrix: $256 \times 192$; \ac{fov}: $280 \times 210$ (with 75\% rectangular \ac{fov}); slab of 16 partitions of \SI{3.5}{\mm} thickness; temporal resolution: \SI{6}{\s}/slab over approximately \SI{5}{\minute}).
A power injector (Medrad, Indianola, USA) is used to provide a bolus injection of Gd-DTPA (Dotarem, Guerbet, Roissy, France) at a dose of \SI{0.2}{\ml} Gd-DTPA/kg of body weight.

These \ac{dce}-\ac{mri} sequences are resampled using the spatial information of the \ac{t2w}-\ac{mri} and missing data are interpolated using a linear interpolation.
The volumes of the \ac{dce}-\ac{mri} dynamic are rigidly registered, to remove any patient motion during the acquisition.
Furthermore, a non-rigid registration is performed between the \ac{t2w}-\ac{mri} and \ac{dce}-\ac{mri} in order to propagate the prostate zones and \ac{cap} ground-truths.
The resampling is implemented in C++ using the Insight Segmentation and Registration Toolkit~\citep{ibanez2005itk}.

\subsection{Implementation}

The implementation of the registration (C++), normalization (Python), and classification pipeline (Python) are publicly available on GitHub\footnote{\url{https://github.com/I2Cvb/lemaitre-2016-nov/tree/master}}~\citep{lemaitre2016github}.
The data used for this work are also publicly available\footnote{\url{https://zenodo.org/record/61163}}~\citep{lemaitre2016dce}.

\section{Experiments and results}\label{sec:experiments}

\subsection{Goodness of model fitting}\label{sec:fit}

%{\color{red} In case that we have issue with $R^2$, we need to provide the AIC since that the model are usually non-linear.}

Parameter estimation of the quantification methods are related to fit a specific model to the \ac{dce}-\ac{mri} data.
Therefore, this section report the goodness of fitting by computing the coefficient of determination $R^2$ such as in Eq.\,\eqref{eq:r2}

\begin{equation}
  R^2 = 1 - \frac{\sum_{t = 1}^{T} (s_t - \hat{s}_t)^2}{\sum_{t = 1}^{T} (s_t - \bar{s})^2} ,
  \label{eq:r2}
\end{equation}

\noindent where $s_t$ and $\hat{s}_t$ are the signal to be fitted and the estimated signal at time $t$, respectively; $\bar{s}$ is the average signal to be fitted.

Mean and standard-deviation of the coefficient of determination $R^{2}$ is reported in Table~\ref{tab:r2} for each quantification model.
Brix, Hoffmann, and Tofts models are fitted with a coefficient $R^{2}$ superior to 0.80.
Additionally, the proposed \ac{pun} model does not seem to fit well the data.
Data normalization improves the coefficient $R^2$ for all the methods apart of the Hoffmann model.
The large standard deviation for this model might imply that there are some cases where the fitting fails.

\subsection{Detection of \acs*{cap} using pharmacokinetic parameters}

To study the potential benefit of our normalization, \ac{cap} are detected at a voxel level using pharmacokinetic parameters estimated from un-normalized and normalized \ac{dce}-\ac{mri} data.
Each individual pharmacokinetic parameter is classified to evaluate their individual discriminative power to detect \ac{cap}.
Therefore, a \ac{rf} classifier is used in conjunction with a \ac{lopo}.
The use of \ac{rf} is motivated since that it leads to the best performance in the state-of-the-art methods~\citep{litjens2014computer,lemaitre2015computer}.
Results are summarized in Table~\ref{tab:resfeats} in terms of \ac{auc}.
Normalization can improve the detection of \ac{cap}; however, the benefit of normalization is more obvious by combining together the pharmacokinetic features of a given model (e.g., $A$, $k_ep$, and $k_el$ for Brix model), as previously done in traditional \ac{cad} system~\citep{lemaitre2015computer}.
For the latter configuration, results are summarized by performing a \ac{roc} analysis and computing the \ac{auc}, as reported in Fig.\,\ref{fig:normpharmarf}.
Quantification using normalized data outperforms quantification using un-normalized data in terms of classification performance apart of Hoffmann and Tofts population-based \ac{aif} models.
The reasons behind the decrease of the \ac{auc} might be related to: (i) a poor fitting as discussed in Sect.\,\ref{sec:fit} (cf., Hoffmann model) and (ii) a small number of patients while estimating some parameters (cf., Tofts model).
The best classification performance are obtained using the semi-quantitative approach with an \ac{auc} of 0.655.

\subsection{Classification of the entire enhanced \acs*{dce}-\acs*{mri} signal}

As stated in the introduction, the quantification methods are extracting a set of parameters characterizing the enhancement \ac{dce}-\ac{mri} signal.
However, this extraction might lead to a loss of information.
This experiment is performed to assess if making use of the whole \ac{dce}-\ac{mri} signal instead of the just the pharmacokinetic parameters can improve the classification performance.
Therefore, each enhanced \ac{dce}-\ac{mri} signal, normalized and un-normalized, is classified using a \ac{rf} classifier in a \ac{lopo} fashion.
The \ac{roc} analysis and \ac{auc} are reported in Fig.\,\ref{fig:rfnormdcesignal}.
Classification without normalization lead to the worst performance, with an \ac{auc} of 0.568.
However, data normalization in conjunction with the use of the whole \ac{dce}-\ac{mri} signal is the strategy which outperforms all others, with an \ac{auc} of 0.666.

%%% Local Variables: 
%%% mode: latex
%%% TeX-master: "../main"
%%% End: 
