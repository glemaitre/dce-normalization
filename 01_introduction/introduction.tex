\section{Introduction}

\Ac{cap} is the second most frequently diagnosed cancer in men,
accounting for 899,000 cases and leading to 258,100 deaths per year~\citep{ferlay2010estimates}.
As highlighted by the PI-RADS Steering Committee, the two main challenges to be addressed are~\citep{weinreb2016pi}:
(i) improving the detection of clinically significant \ac{cap} and
(ii) increasing confidence in benign or dormant cases and therefore avoiding unnecessary invasive medical exams.
In this regard, \ac{mpmri} is frequently used to build robust \ac{cad} systems to detect, localize, and grade \ac{cap}.
In general, \ac{cad} systems are based on \ac{mpmri} which potentially combines several of the following modalities~\citep{lemaitre2015computer}: \ac{t2w}-\ac{mri}, \ac{dce}-\ac{mri}, \ac{adc} maps, and \ac{mrsi}.

In \ac{dce}-\ac{mri}, a contrast media is injected intravenously and a set of images is acquired over time.
Consequently, each voxel in an image corresponds to a dynamic signal that is related to both contrast agent concentration and the vascular properties of the tissue.
Therefore, changes in the enhanced signal allows for the
discrimination of healthy tissues from \ac{cap} tissues.
In fact, these properties are automatically extracted using quantitative or semi-quantitative approaches~\citep{lemaitre2015computer}.

\emph{Quantitative} approaches uses pharmacokinetic modelling based on a bicompartment model, namely Brix~\citep{brix1991pharmacokinetic} and Tofts~\citep{tofts1995quantitative} models.
The parameters of the Brix model are inferred by assuming a linear relationship between the media concentration and the \ac{mri} signal intensity.
However, this assumption has been shown to lead to inaccurate
estimations of the pharmacokinetic parameters~\citep{heilmann2006determination}.
In contrast, the Tofts model requires the conversion of \ac{mri}
signal intensity to concentration, which becomes a non-linear
relationship using a specific equation of \ac{mri} sequences (e.g., FLASH sequence).
Tofts modeling, however, is highly complex~\citep{gliozzi2011phenomenological}.
Achieving the conversion using the non-linear approach requires the
acquisition of a T$_1$ map which is not always possible during clinical examination.
Additionally, the parameter calculation requires the \ac{aif} which is challenging to measure and can also lead to an inaccurate estimation.

\emph{Semi-quantitative} approaches are mathematical rather than pharmacokinetic because no pharmacokinetic assumptions regarding the relationship between the \ac{mri} signal and the contrast agent are made~\citep{huisman2001accurate,gliozzi2011phenomenological}.
These methods are advantageous because they do not require any knowledge of the \ac{mri} sequence or any conversion from signal intensity to concentration.
However, they present some limitations: the heuristic approach
proposed by \citeauthor{huisman2001accurate} requires an initial
estimate of the standard deviation of the signal noise and some manual tuning.

Nevertheless, all of the presented methods suffer from the following two major drawbacks:
(i) inter-patient variability and (ii) loss of information.
The inter-patient variability is mainly due to the acquisition process
and consequently leads to generalization issues in applying a machine learning algorithm.
All previous methods extract few discriminative parameters to describe the \ac{dce}-\ac{mri} signal which might lead to a loss of information.
%(i) the inter-patient variability of the data lead to a variation of the parameters estimated and subsequently to poor classification performance while designing \ac{cad} systems, and
%(ii) only few parameters are used to characterize the dynamic signal implying that some information are discarded.

In this work, we propose a fully automatic normalization method for \ac{dce}-\ac{mri} that reduces the inter-patient variability of the data.
The benefit and simplicity of our approach will be demonstrated by
classifying the whole normalized \ac{dce}-\ac{mri} signal and
comparing it with state-of-the-art quantitative and semi-quantitative methods.
Additionally, we will show that using this normalization approach in conjunction with the quantitative methods improves the classification performance of most of the models.
We also propose a new clustering-based method to discern enhanced
signals from the arteries that can, later be used to estimate an
\ac{aif} and provide an alternative approach to estimate the parameters of the semi-quantitative model proposed by~\cite{huisman2001accurate}.

% The benefit of our approach will be shown while using quantitative and semi-quantitative approaches.
% Additionally, we show that using the whole normalized \ac{dce}-\ac{mri} signal is preferable to quantitative and semi-quantitative methods, leading to the best classification performance.

The paper is organized as follows:
Section~\ref{sec:norm} details our normalization strategy for the \ac{dce}-\ac{mri} data.
Quantitative and semi-quantitative methods are summarized in Sect.\,\ref{sec:stateart} with insights about their implementations.
Section~\ref{sec:materials} provides information about the dataset
used and the provided source code.
Experiments and results that address the previously stated challenges
are reported in Sect.\,\ref{sec:experiments} and discussed in Sect.\,\ref{sec:discussions}, followed by a concluding section.
%Section~\ref{sec:methods} outlines our normalization strategy (Section~\ref{sec:norm}) as well as specificity regarding the state-of-the-art methods used for comparison (Section~\ref{sec:stateart}).
%The dataset, experiments, and results are reported in Section~\ref{sec:experiments} while discussed in Section~\ref{sec:discussions} followed by a concluding section.

%%% Local Variables:
%%% mode: latex
%%% TeX-master: "../main"
%%% End:
