%%%%%%%%%%%%%%%%%%%%%%%%%%%%%%%%%%%%%%%%%
% Thin Formal Letter
% LaTeX Template
% Version 1.11 (8/12/12)
%
% This template has been downloaded from:
% http://www.LaTeXTemplates.com
%
% Original author:
% WikiBooks (http://en.wikibooks.org/wiki/LaTeX/Letters)
%
% License:
% CC BY-NC-SA 3.0 (http://creativecommons.org/licenses/by-nc-sa/3.0/)
%
%%%%%%%%%%%%%%%%%%%%%%%%%%%%%%%%%%%%%%%%%

% ----------------------------------------------------------------------------------------
% DOCUMENT CONFIGURATIONS
% ----------------------------------------------------------------------------------------

\documentclass{letter}

% Adjust margins for aesthetics
\addtolength{\voffset}{-0.5in}
%\addtolength{\hoffset}{-0.3in}
\addtolength{\textheight}{2cm}

\usepackage[letterpaper, margin=1.5in]{geometry}

%\longindentation=0pt % Un-commenting this line will push the closing "Sincerely," to the left of the page

% ----------------------------------------------------------------------------------------
% YOUR NAME & ADDRESS SECTION
% ----------------------------------------------------------------------------------------

\signature{Guillaume Lema\^itre} % Your name for the signature at the bottom

\address{Parietal team, Inria \\ CEA, Universit\'e Paris-Saclay \\ 1 Rue
  Honor\'e d'Estienne d'Orves \\ 91120 Palaiseau \\ France} % Your address
% and phone number

% ----------------------------------------------------------------------------------------

\begin{document}

% ----------------------------------------------------------------------------------------
% ADDRESSEE SECTION
% ----------------------------------------------------------------------------------------

\begin{letter}{Nicholas Ayache \\ Co-editor in Chief \\ Medical Image Analysis} % Name/title of the addressee

  % ----------------------------------------------------------------------------------------
  % LETTER CONTENT SECTION
  % ----------------------------------------------------------------------------------------

  \opening{
    \textit{Subject: MEDIA-D-16-00507 - Response to reviewers}\\ \\
    \textbf{Dear Sir or Madam,}}

  We would like to thank the reviewers for their pertinent
  remarks that have increased the quality of the manuscript. Whenever
  possible, the comments of the reviewers have been addressed and
  highlighted in the article.

  Please, find below a point-by-point response to each of the issues raised
  by the reviewers.

  Reviewer \#1:
  \begin{enumerate}
  \item \textbf{In a typical DCE based analysis, the contrast
      enhanced images are rigidly registered to T1w images for
      intensity normalization. Typically for multi-parametric MRI
      based CAD system for prostate cancer detection T1w images
      are present. This is useful in differentiating cancers from
      blood hemorrhage following prostate biopsy. Please discuss
      why a pre or a post contrast T1w image is missing from the
      clinical trial that will be using DCE images.}\\
    Maybe it was not entirely clear from the text but the DCE sequence
    used in this work is based on T1w. In other words, the
    pre-contrast acquisition corresponds to the first volumes in the
    sequence before the contrast injection. To avoid confusion, this
    has been clarified in the Material section (p.~15/l.~275).
    Additionally, the risk of blood hemorrhage is discarded since the
    biopsies are carried out after the MRI acquisition.
  \item \textbf{The other advantage of normalizing to T1w images are
      patient specific information will be preserved. Different
      patients may have different cancer lesions of different grades
      (having different kinetic expression in DCE) and normalizing all
      patients to the same space may result in loss of
      information. Please comment.}\\
    Our normalization is designed to align the \emph{mean kinetic
      expression}. The mean kinetic expression contains cancer and
    non-cancer kinetic information. Assuming that the amount of cancer
    information (voxels) is much lower than the normal tissue, the use
    of this normalization allows to minimize undersired
    patient-specific (overall-intensity, mean contrast, uptake, ...)
    without affecting the lesion information.
  \item \textbf{T1w normalization will be probably be more robust to
      change in magnet, machine, reconstruction software and
      acquisition parameters across different institutions as the
      objective is normalization to be patient and institution
      specific. Please comment how the presented model may be affected
      in a multi-institutional study.}\\
    In line with the previous comment on the use of T1w, we want to
    clarify that DCE is based on T1w sequence.
    Moreover, our method does not rely on any manufacturer/machine-specific settings.
    Subsequently, our approach should work on any type
    of devices, enabling multi-institutional studies. We included
    these remarks in our future work section in p.~23/l.~396.
  \item \textbf{Please discuss how the improvement in prostate cancer
      detection in DCE-MRI impacts the current clinical CAD
      pipeline. According to PIRADS v2 guidelines more emphasis is
      laid upon T1w, T2w and DWI compared to DCE images. Experienced
      radiologists can detect prostate cancers in peripheral zone with
      a high degree of accuracy (higher than 0.6 AUC). The transition
      zone lesions are more difficult to distinguish. Please comment
      how the model performs in detection of transition zone
      lesions. Further please comment how this model adds value to
      already established clinical pipeline of detecting prostate
      cancer (PIRADS).}\\
    We have included a new stratified analysis, separating PZ and CG,
    which showed an increase of the classification performance in CG after
    applying our
    normalization method as depicted in Fig.\,8 p.~22.
    The PIRADS v2 emphasizes that DCE-MRI has to be included in
    mp-MRI acquisition to detect small lesions. However, it also warns
    that the benefit is modest. Therefore, the impact of an
    improvement of detection in DCE-MRI can only be evaluated in a
    fully mp-MRI CAD system which is beyond the scope of this paper.
    Indeed, we recently showed, --- Lemaitre,
    Guillaume. Computer-Aided Diagnosis for Prostate Cancer using
    Multi-Parametric Magnetic Resonance Imaging. Diss. Universit\'e de
    Bourgogne; Universitat de Girona, 2016;
    Chapt.\,6/p.~150/Table~6.10 ---, that DCE-MRI features are
    considered important and the combination with other modalities
    will lead to better classification results.
    However, in our humble opinion, including those additional results
    will clutter the paper.
  \item \textbf{Comparing the results of semi-automatic approach and
      the presented approach of curve fitting using the entire curve
      the results are not significantly different (0.65 AUC vs 0.66
      AUC). The presented model is however automatic which would
      reduce inter-rater variance in a semi-automatic method. Please
      present a discussion comparing these two results.}\\
    It is important to notice that for both approaches, ---
    semi-quantitative and entire enhanced signal ---, the results
    are obtained using our normalization approach, showing the importance of
    this step. We added a discussion in p.~23/l.~388 regarding the
    semi-quantitative approach and entire enhanced signal
    classification. The semi-quantitative approach reduces the number
    of features to analyze, speeding up the classification. However,
    it relies on the curve fitting which is also time consuming and
    might fail.
  \item \textbf{Use of single feature for a random forest classifier
      (Table 2) may be detrimental for the classifier as random
      sub-sampling of the features are not possible. Please present
      the comparison with some linear classifiers like linear SVM.}\\
    A classification using a linear SVM has been added in additional
    material.
    Unlike a linear SVM, RF allows to find multiple thresholds,
    leading to a much complex decision boundary for this
    feature. Bootstrapping the samples for each tree allows to cope
    with the over-fitting of each decision trees. Results with SVM
    show no particular improvement in comparison with RF. Therefore,
    we do not
    think that including a linear SVM will be beneficial for the paper
    as are the results provided in the additional material showing.
  \end{enumerate}

  Reviewer \#4:
  \begin{enumerate}
  \item \textbf{The title is misleading and not very scientific and I
      would recommend to remove the second part and reword the first
      part to cover the contents of the paper.}\\
    We are not against changing the title but we would like to take
    advantage of this revision to get the opinion of all reviewers
    regarding this matter.
  \item \textbf{The use of the English language needs some attention:
      get it all proof read.}\\
    As suggested, the manuscript has been corrected by an English editing
    service to ensure a good quality writing. A certificate of the
    service can be provided upon request.
  \item \textbf{What guided biopsy was used: US?}\\
    The guided biopsy was performed with a TRUS. We added this
    statement in p.~15/l.~275.
  \item \textbf{The number of cases seems low. This is mentioned as
      possibly affecting the results and that the limitations might be
      based on the sample size. At the end of section 5 one could draw
      the conclusion that they should do a more extensive study.}\\
    The reviewer is right with the comment, unfortunately due to the
    difficulty on acquiring and annotating this kind of data, we
    cannot provide a larger database at this time. However, a
    statement has been added in the discussions p.~23/l.~396,
    describing this limitation of the study.
  \item \textbf{How would this translate to other types of MRI
      systems: it would be good to see experimental results on this or
      at least some discussion.}\\
    Refer to reviewer \#1 / comment 3 since this is a shared remark.
  \item \textbf{The disappointing performance and large standard
      deviation of the PUN model should be investigated and
      explained.}\\
    A discussion is added in p.~17/l.~320. The original published
    model lacks of
    degrees of freedom. Two additional parameters will
    allow for a better fit. However, we used the original formulation
    in our experiments.
  \item \textbf{There should be a detailed discussion on the results
      to indicate which differences are statistically
      significant. ``Significant'' is mentioned in the conclusions, but
      there is no evidence for this.}\\
    We added a Wilcoxon significance test - see Fig.\,7. Now, the term
    significant is only used for $p<0.05$.
  \item \textbf{The overall results seem poor and this could be
      further discussed.}\\
    It is difficult to compare with the state-of-the-art methods since
    the dataset used are different and the reported results in those
    studies are usually linked to a mp-MRI CAD instead of only
    DCE. Nevertheless, our results are in the same range than Litjens,
    Geert, et al. "Computer-aided detection of prostate cancer in
    MRI." IEEE transactions on medical imaging 33.5 (2014):
    1083-1092 regarding the classification performance of the Tofts
    parameters.
  \item \textbf{Sections 5 and 6 are both disappointingly short.}
    After including the remarks from the reviewers, this section has
    become much longer.
    % However,
    % we believed that there is no need for repeating information
    % already presented in the experiment section or give hypothesis
    % which cannot be verified.
  \item \textbf{Some journals references incomplete. Gliozzi et al. is
      not a single page. Check: Schabel, Shanbhag, Zhu.}\\
    The references have been corrected.
  \end{enumerate}

  Reviewer \#5:
  \begin{enumerate}
  \item \textbf{There are however several issues with the paper In the
      current state. Using quantitative parameters obtained from
      pharmacokinetic models as an end-point comparison indeed may be
      important as the models provide physiological information
      related to capillary permeability, blood fraction and
      interstitial space volume, etc.. However I am not really
      convinced by the detailed results presented in Table 2, as no
      statistical test was performed when the classification
      performances are higher with normalized data, and for some of
      the models the results are worst. For me, the benefits of the
      method are not fully demonstrated in terms of classification
      performance using the models. How these parameters are modified
      with the normalization?}\\
    This is important to emphasize that our goal was to propose a
    normalization method to allow classification of the entire
    enhanced signal which should perform as good as the current
    models. For the sake of completeness, we apply the same
    normalization on the different
    pharmacokinetic models and, as mentioned by the reviewer ---
    the results are not significantly better as the model parameters
    might have a different interpretation after the normalization is
    applied.
    Indeed, these experiments were carried out to investigate the
    benefit or not of the normalization with the current models.
  % \item \textbf{Although quantitative parameters obtained from
  %     pharmacokinetic models applied to DCE-MRI series have
  %     demonstrated in general superior performance to differentiate
  %     cancerous from normal tissue prostate in comparison to simpler
  %     enhancement analysis such as semi-quantitative or qualitative
  %     parameters, there are still several differences in results
  %     across the studies in the literature. There are some papers
  %     aimed at defining normal values for some of the pharmacokinetic
  %     parameters. The authors may refer to the paper Sanz-Requena et
  %     al. Dynamic contrast-enhanced case-control analysis in
  %     3T MRI of prostate cancer can help to characterize tumor
  %     aggressiveness, European Journal of Radiology, 85(11), November 2016,
  %     pp. 2119-2126, for a detailed list of typical values and difficulties
  %     of the models}\\
  \item \textbf{Some of the benefits or improvements may be
      demonstrated before any further classification step. The method
      could be useful when performing population analysis with the raw
      data, previous to defining a model.}\\
    We do not really see which analysis can be performed to evaluate
    the benefits of our methods before the classification step. A study of
    the variances with and without normalization could be carried out but
    an evaluation after classification seems equivalent and a good manner to
    evaluate the benefit of the normalization. In fact, these are the
    results reported in Fig.\,6(c).
  \item \textbf{Figure 1 is not pertinent for the goal of this paper.
      It is not clear how is represented the interindividual variation in
      figure 2?  In Fig 2a the PDFs are concatenated in time to obtain a
      heatmap representation, but in 2 b-c  the differences across
      individuals does not appear.}\\
    Figure 1 is used to illustrate inter-patients variation in T2w-MRI
    and to highlight the need of a different normalization approach
    than the one needed in T2w. We have clarified in the caption that
    Fig.\,1 is related to T2w, and if requested by the reviewers,
    Figure 1 could be removed. We improve the Fig.\,2. Figure~2(a)
    presents only the process to build the
    heatmap. In Fig.\,2(b)-2(c), we added parameters which highlight
    more clearly the variations and added their corresponding values
    in caption.
  \item \textbf{It is not clear the rational of equation 1, as the
      parameters are rather empirical.  There are no other ways to
      smooth the heatmap, for example through a diffusion process,
      solving a PDE equation? Or by registering through the series?}\\
    There is probably a confusion. We do not try to smooth the heatmap
    but we search a smooth estimate. Equation 1 represents the weight
    of the graph and is a standard way to formalize graph-based
    problem in image processing which often includes a data driven
    term and a smoothing term.
  \item \textbf{The theoretical foundations of equation 3 are not
      clear. Other registration methods for time signas may have been
      included.}\\
    While other registrations could have been used,
    using a rigid registration directly affect the variations
    $\alpha_i$ and $\Delta_t$. Non rigid registration is deemed not
    necessary as the main parameters that we propose to recover are
    translations and scaling. Moreover, this kind of registration
    might induce deformation artifacts which could impact on the
    overall contrast enhancement over time.
  \item \textbf{Other question that may be raised is the influence of
      the equation parameters (tau, alfa..) in the pharmacokinetic
      models. Do they change with f(t)?}\\
    We think that this question is related to the results presented in
    Sect.\,4.1
  \item \textbf{A thorough review of the structure and in general an
      improvement of the paper in terms of writing may be helpful to
      fully understand the message, the rational and the proposed
      methodology.}\\
    As suggested, the manuscript has been corrected by an English editing
    service to ensure a good quality writing.
  \end{enumerate}

  Thank you for your time and consideration.

  I look forward to your reply.

  \vspace{2\parskip} % Extra whitespace for aesthetics
  \closing{Sincerely,}
  \vspace{2\parskip} % Extra whitespace for aesthetics

  % ----------------------------------------------------------------------------------------

\end{letter}

\end{document}
