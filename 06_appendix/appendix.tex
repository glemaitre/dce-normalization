\appendix
\section{Conversion from FLASH signal to media concentration}\label{app:signaltoconc}

In this appendix, we show the demonstration used to extract the agent concentration from the \ac{mri} signal.

The signal equation in FLASH sequence~\citep{haase1986flash} is defined as:

\begin{equation}
  S(t) = S_{eq} \sin \alpha \cdot \frac{1 - \exp\left(-TR\left(R_{10} + r_1 C(t)\right)\right)}{1 - \cos \alpha \cdot \exp\left(-TR\left(R_{10} + r_1 C(t)\right)\right)} ,
  \label{eq:app:flash}
\end{equation}

\noindent where $S(t)$ is the \ac{mri} signal, $S_{eq}$ is the maximum signal amplitude of the spoiled gradient at the \ac{te} which is proportional to the \ac{pd}, $\alpha$ is the flip angle, $TR$ is the \acf{tr}, $R_{10}$ is the pre-contrast tissue relaxation time also equal to $\frac{1}{T_{10}}$, $r_1$ is the relaxitivity coefficient of the contrast agent, and $C(t)$ is the media concentration.

Therefore, the pre-contrast signal prior to bolus injection of the media is defined as:

\begin{equation}
  S_0 = S_{eq} \sin \alpha \cdot \frac{1 - \exp\left(-TR \cdot R_{10}\right)}{1 - \cos \alpha \cdot \exp\left(-TR \cdot R_{10}\right)} .
  \label{eq:app:precontrast}
\end{equation}

To simplify the demonstration, let us define:

\begin{align}
  a &= \exp(-TR \cdot R_{10}) , \\
  b &= \exp(-TR \cdot r_1 C(t)) .
\end{align}

Let us define:

\begin{align}
  S^{*} &= \frac{S_0}{S_{eq} \sin \alpha} , \\
  &= \frac{1 - a}{1 - a \cos \alpha} .
\end{align}

Thus,

\begin{align}
  S^{*}\frac{S(t)}{S_0} &= \frac{S_0}{S_{eq}\sin \alpha} \frac{S(t)}{S_0} , \\
  &= \frac{1 - a b}{1 - a b \cos \alpha} .
\end{align}

Now, let us define:

\begin{align}
  \frac{1 - \cos \alpha \cdot S^{*}\frac{S(t)}{S_0}}{1 - S^{*}\frac{S(t)}{S_0}} &= \frac{ 1 - \cos \alpha\left(\frac{1 - a b}{1 - a b \cos \alpha}\right)} {1 - \frac{1 - a b}{1 - a b \cos \alpha}} , \\
  &= \frac{1 - a b \cos \alpha - \cos \alpha(1 - a b)}{1 - a b \cos \alpha - (1 - a b)} , \\
  &= \frac{1 - a b \cos \alpha - \cos \alpha + a b \cos \alpha}{1 - a b \cos \alpha - 1 + a b} , \\
  &= \frac{1 - \cos \alpha}{ ab (1 - \cos \alpha)}, \\
  &= \frac{1}{a b}.
\end{align}

Thus,

\begin{equation}
  -TR \cdot R_{10} -TR \cdot r_1 C(t) = \ln\left( \frac{1 - \cos \alpha \cdot S^{*}\frac{S(t)}{S_0}}{1 - S^{*}\frac{S(t)}{S_0}} \right) .
\end{equation}

Therefore,

\begin{equation}
  C(t) = \frac{1}{TR \cdot r_1} \ln\left( \frac{1 - \cos \alpha \cdot S^{*}\frac{S(t)}{S_0}}{1 - S^{*}\frac{S(t)}{S_0}} \right) - \frac{R_{10}}{r_1} .
\end{equation}


%%% Local Variables: 
%%% mode: latex
%%% TeX-master: "../main"
%%% End: 
